
\documentclass[12pt,spanish]{report}
\usepackage[spanish]{babel}
\selectlanguage{spanish}
\usepackage[utf8]{inputenc}
\usepackage{blindtext}
\usepackage{enumitem}
\usepackage[a4paper,left=3cm,right=2cm,top=2.5cm,bottom=2.5cm]{geometry}

\title{Sistema de gestión de transportes \\  \small{para Envíamelo S.A.}}
\date{\today}
\author{Carlos Sánchez Páez \\ José Miguel Pelegrina Pelegrina \\ José Baena Cobos}
\begin{document}

\maketitle
En este documento se especifican la requisitos del sistema de gestión de la empresa Envíamelo S.A.

\section*{Objetivos}

Este proyecto tiene como objetivo desarrollar un software para sustituir al actual sistema de gestión de la empresa de transporte Envíamelo S.A. Ésta dispone de una flota de furgonetas para la distribución de paquetes por todo el territorio español. En cada capital de provincia hay una oficina y un almacén para la carga y descarga de los paquetes que van a ser transportados de un almacén de origen a otro de destino. Además, cada almacén dispone de un conjunto de pequeñas furgonetas que se encargan de la recogida de paquetes de los usuarios que quieren enviar un paquete, así como de su reparto al destinatario final. \\
Este nuevo sistema busca agilizar los trámites tanto en la recogida, como el envío y recepción del paquete; a la vez que facilitar al cliente la gestión de sus pedidos. Con esto se pretende aumentar el margen de beneficios de la empresa.\\ \\

Objetivos orientados al cliente: 
\begin{description}[align=right,labelwidth=3cm]
\item [OBJ-1] Facilitar a los clientes la gestión de sus pedidos de forma rápida, fácil y eficiente.
\item [OBJ-2] Garantizar un buen servicio posventa, permitiendo al cliente en todo momento conocer el estado de su pedido.
\end{description}
Objetivos orientados a la empresa:
\begin{description}[align=right,labelwidth=3cm]
\item [OBJ-3] Gestionar los almacenes.
\item [OBJ-4] Optimizar la ruta de transporte y reparto para poder tramitar mayor volumen de paquetes.
\item [OBJ-5] Gestionar la disponibilidad de los conductores de furgonetas y del personal de almacén. 
\item [OBJ-6] Conocer la opinión de los clientes para poder mejorar el servicio.
\end{description}
\end{document}