\subsection{Carlos Sánchez Páez}

%%%%%%%%%%%ENVIAR PAQUETE%%%%%%%%%%%%%%%%
\begin{table}[H]
\centering
\begin{tabular}{|m{3cm}|m{4cm}|m{2cm}|m{2cm}|m{2cm}|m{1cm}|}
\hline
\textbf{Caso de uso} &  \multicolumn{4}{m{8cm}|}{Enviar paquete} \vline &  \cellcolor{gray!40}CU1 \\
\hline
\textbf{Actores} & \multicolumn{5}{m{8cm}|}{Socio o emisor} \\
\hline
\textbf{Tipo} & \multicolumn{5}{m{8cm}|}{Esencial} \\
\hline
\textbf{Referencias} &\multicolumn{3}{m{4cm}|}{-} & \multicolumn{2}{m{4cm}|}{-} \\
\hline
\textbf{Precondición} & \multicolumn{5}{m{8cm}|}{-} \\
\hline
\textbf{Postcondición} & \multicolumn{5}{m{8cm}|}{-} \\
\hline
\textbf{Autor} & Carlos Sánchez Páez & \textbf{Fecha} & 19/03/18 & \textbf{Versión} & 3.0 \\
\hline
\end{tabular}

\vspace{1cm}

\begin{tabular}{|m{16.2cm}|}
\hline
\textbf{Propósito} \\
\hline
Que el usuario solicite la recogida (o bien lleve a una oficina) el paquete que quiere enviar. \\
\hline
\end{tabular}

\vspace{1cm}

\begin{tabular}{|m{16.2cm}|}
\hline
\textbf{Resumen} \\
\hline
El usuario entrega el paquete a la empresa para que ésta lo lleve a su destinatario. \\
\hline
\end{tabular}

\vspace{1cm}

\begin{tabular}{|m{4pt}|m{7.33cm}|m{4pt}|m{7.33cm}|}
\hline
\multicolumn{4}{|c|}{\textbf{Curso normal}} \\
\hline
\textbf{1} & El cliente elige la tarifa que más le conviene. & \textbf{2} & Si es un socio, se identifica como tal en el sistema. \\
\hline
\textbf{3} & El sistema solicita la introducción de los datos necesarios (destinatario, peso del bulto, medidas, franja horaria de entrega, etc.) & \textbf{4} & El cliente acuerda la recogida del bulto o bien lo lleva él mismo a la oficina correspondiente. \\
\hline
\end{tabular}

\vspace{1cm}

\begin{tabular}{|m{10pt}|m{7.15cm}|m{10pt}|m{7.15cm}|}
\hline
\multicolumn{4}{|c|}{\textbf{Cursos alternos}} \\
\hline
\textbf{2a} & Si el cliente desea hacerse socio, el sistema le ofrece acceso al procedimiento de \emph{alta}. & \textbf{3a} & Si el cliente no posee los datos físicos del bulto, se le ofrece la posibilidad de que su medición y pesaje se realice en el momento de la recogida.  \\
\hline
\end{tabular}

\vspace{1cm}

\begin{tabular}{|m{3.72cm}|m{3.72cm}|m{3.72cm}|m{3.72cm}|}
\hline
\multicolumn{4}{|c|}{\textbf{Otros datos}} \\
\hline
\textbf{Frecuencia esperada} & 20 veces al día & \textbf{Rendimiento} & - \\
\hline
\textbf{Importancia} & Alta & \textbf{Urgencia} & Inmediatamente \\
\hline
\textbf{Estado} & En construcción & \textbf{Estabilidad} & Alta \\
\hline
\end{tabular}

\vspace{1cm}

\begin{tabular}{|m{16.2cm}|}
\hline
\textbf{Comentarios} \\
\hline
La frecuencia esperada dependerá de la fecha, siendo más alta en fechas críticas como navidad. \\
\hline
\end{tabular}

\caption{Enviar un paquete}

\end{table}
%%%%%%%%%%%%%%%%%%%%%%%%%%%%%%%%%%%%%%%%%%%

%%%%%%%%%%DIRIGIR EMPLEADOS%%%%%%%%%%%%%%%


\begin{table}[H]
\centering
\begin{tabular}{|m{3cm}|m{4cm}|m{2cm}|m{2cm}|m{2cm}|m{1cm}|}
\hline
\textbf{Caso de uso} &  \multicolumn{4}{m{8cm}|}{Dirigir a los empleados} \vline &  \cellcolor{gray!40}CU4 \\
\hline
\textbf{Actores} & \multicolumn{5}{m{8cm}|}{\begin{enumerate}
\item Encargado de oficina.
\item Encargado de almacén.
\item Conductor.
\end{enumerate}} \\
\hline
\textbf{Tipo} & \multicolumn{5}{m{8cm}|}{Vital} \\
\hline
\textbf{Referencias} &\multicolumn{5}{m{8cm}|}{CU3} \\
\hline
\textbf{Precondición} & \multicolumn{5}{m{8cm}|}{Tiene que haber al menos un paquete que deba ser entregado.} \\
\hline
\textbf{Postcondición} & \multicolumn{5}{m{8cm}|}{-} \\
\hline
\textbf{Autor} & Carlos Sánchez Páez & \textbf{Fecha} & 19/03/2018 & \textbf{Versión} & 2.0 \\
\hline
\end{tabular}

\vspace{1cm}

\begin{tabular}{|m{16.2cm}|}
\hline
\textbf{Propósito} \\
\hline
Asignar el trabajo correspondiente a cada conductor y mozo de almacén. \\
\hline
\end{tabular}

\vspace{1cm}

\begin{tabular}{|m{16.2cm}|}
\hline
\textbf{Resumen} \\
\hline
El encargado de oficina distribuye el trabajo que debe realizar cada conductor y cada mozo de almacén. \\
\hline
\end{tabular}

\vspace{1cm}

\begin{tabular}{|m{4pt}|m{7.33cm}|m{4pt}|m{7.33cm}|}
\hline
\multicolumn{4}{|c|}{\textbf{Curso normal}} \\
\hline
\textbf{1} & El encargado de oficina solicita al sistema que realice un trazado de la ruta según el origen y el destino de los paquetes. & \textbf{2} & El sistema devuelve el trazado \\
\hline
\textbf{3} & El encargado de oficina asigna a los conductores un vehículo en concreto para que realice su función. & \textbf{4} & El encargado de oficina atribuye a los mozos de almacén el trabajo a realizar (categorización, carga/descarga, etc.) en cada franja horaria. \\
\hline
\end{tabular}

\vspace{1cm}

\begin{tabular}{|m{10pt}|m{7.15cm}|m{10pt}|m{7.15cm}|}
\hline
\multicolumn{4}{|m{16.2cm}|}{\textbf{Cursos alternos}} \\
\hline
\multicolumn{4}{|m{16.2cm}|}{\textbf{-}} \\
\hline
\end{tabular}

\vspace{1cm}

\begin{tabular}{|m{3.72cm}|m{3.72cm}|m{3.72cm}|m{3.72cm}|}
\hline
\multicolumn{4}{|c|}{\textbf{Otros datos}} \\
\hline
\textbf{Frecuencia esperada} & Alta (a diario) & \textbf{Rendimiento} & - \\
\hline
\textbf{Importancia} & Extrema & \textbf{Urgencia} & Máxima \\
\hline
\textbf{Estado} & En desarrollo & \textbf{Estabilidad} & Alta \\
\hline
\end{tabular}

\vspace{1cm}

\begin{tabular}{|m{16.2cm}|}
\hline
\textbf{Comentarios} \\
\hline
- \\
\hline
\end{tabular}

\caption{Dirigir a los empleados}

\end{table}