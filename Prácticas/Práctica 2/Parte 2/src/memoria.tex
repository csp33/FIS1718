\documentclass[12pt,spanish]{article}
\usepackage[spanish]{babel}
\usepackage{graphicx}
\usepackage{multirow}
\usepackage{float}
\usepackage{enumitem}
\usepackage[hidelinks]{hyperref}
\usepackage{array}
\graphicspath{ {../../../../LaTeX/img/} {../diagramas/} {./img/} {../../LaTeX/img/}}
\usepackage[table]{xcolor}
\selectlanguage{spanish}
\usepackage[utf8]{inputenc}
\usepackage{graphicx}
\usepackage[ddmmyy]{datetime}
\usepackage[a4paper,left=3cm,right=2cm,top=2.5cm,bottom=2.5cm]{geometry}
\makeindex

\begin{document}
\begin{titlepage}

\newlength{\centeroffset}
\setlength{\centeroffset}{-0.5\oddsidemargin}
\addtolength{\centeroffset}{0.5\evensidemargin}
\thispagestyle{empty}

\noindent\hspace*{\centeroffset}\begin{minipage}{\textwidth}

\centering
\includegraphics[width=0.9\textwidth]{logo_ugr.jpg}\\[1.4cm]

\textsc{ \Large Fundamentos de Ingeniería del Software\\[0.2cm]}
\textsc{GRADO EN INGENIERÍA INFORMÁTICA}\\[1cm]

{\Huge\bfseries Práctica 2. Modelos de casos de uso\\
}
\noindent\rule[-1ex]{\textwidth}{3pt}\\[3.5ex]
{\large\bfseries Segunda parte}
\end{minipage}

\vspace{2.5cm}
\noindent\hspace*{\centeroffset}
\begin{minipage}{\textwidth}
\centering

\textbf{Autores}\\ {José Baena Cobos \\ José Miguel Pelegrina Pelegrina\\Carlos Sánchez Páez}\\[2.5ex]
\includegraphics[width=0.3\textwidth]{etsiit_logo.png}\\[0.1cm]
\vspace{1.5cm}
\includegraphics[width=0.2\textwidth]{lsi.png}\\[0.1cm]
\vspace{1cm}
\textsc{Escuela Técnica Superior de Ingenierías Informática y de Telecomunicación}\\
\vspace{1cm}
\textsc{Curso 2017-2018}
\end{minipage}
\end{titlepage}
\tableofcontents
\thispagestyle{empty}
\listoffigures
\listoftables
\setcounter{page}{1}
%%%%%%%%%%%%%%%%%%%%%%%%Comienzo del documento%%%%%%%%%%%%%%%%%%%%%%%%%%%%%%%

\section{Diagrama de actividad}


\section{Descripción de casos de uso a nivel extendido}

\newcounter{contadorCU}
\setcounter{contadorCU}{1}


%%%%ALTA EMPLEADO%%%%

\begin{table}[H]
\centering
\begin{tabular}{|m{3cm}|m{4cm}|m{2cm}|m{2cm}|m{2cm}|m{1cm}|}
\hline
\textbf{Caso de uso} &  \multicolumn{4}{m{8cm}|}{Alta empleado} \vline &  \cellcolor{gray!40}CU\arabic{contadorCU}  \stepcounter{contadorCU}
\\
\hline
\textbf{Actores} & \multicolumn{5}{m{8cm}|}{Empleado y oficinista} \\
\hline
\textbf{Tipo} & \multicolumn{5}{m{8cm}|}{Real} \\
\hline
\textbf{Referencias} &\multicolumn{5}{m{8cm}|}{-} \\
\hline
\textbf{Precondición} & \multicolumn{5}{m{8cm}|}{-} \\
\hline
\textbf{Postcondición} & \multicolumn{5}{m{8cm}|}{El nuevo empleado podrá acceder al sistema mediante un login} \\
\hline
\textbf{Autor} & Carlos Sánchez Páez & \textbf{Fecha} & 08/04/2018 & \textbf{Versión} & 2.0 \\
\hline
\end{tabular}

\vspace{1cm}

\begin{tabular}{|m{16.2cm}|}
\hline
\textbf{Propósito} \\
\hline
Poder incorporar a un nuevo empleado al sistema. \\
\hline
\end{tabular}

\vspace{1cm}

\begin{tabular}{|m{16.2cm}|}
\hline
\textbf{Resumen} \\
\hline
Un nuevo empleado pasará a formar parte de la empresa. Podrá acceder al sistema con sus credenciales y tendrá unos privilegios determinados. \\
\hline
\end{tabular}

\vspace{1cm}

\begin{tabular}{|m{4pt}|m{7.33cm}|m{4pt}|m{7.33cm}|}
\hline
\multicolumn{4}{|c|}{\textbf{Curso normal}} \\
\hline
\textbf{1} & El oficinista pide al nuevo empleado sus datos y los introduce en el sistema. & \textbf{2} & El sistema comprueba que los datos sean válidos y no existan ya. \\
\hline
\textbf{3} &  & \textbf{4} & El sistema imprime los credenciales de acceso del nuevo empleado. \\
\hline
\textbf{5} & El oficinista da los credenciales al empleado & \textbf{6} &  \\
\hline
\end{tabular}

\vspace{1cm}

\begin{tabular}{|m{10pt}|m{7.15cm}|m{10pt}|m{7.15cm}|}
\hline
\multicolumn{4}{|m{16.2cm}|}{\textbf{Cursos alternos}} \\
\hline
\multicolumn{4}{|m{16.2cm}|}{\textbf{2a) Los datos no son correctos}} \\
\hline
\multicolumn{4}{|m{16.2cm}|}{El sistema informa del error y vuelve a solicitar los datos.} \\
\hline
\end{tabular}

\vspace{1cm}

\begin{tabular}{|m{3.72cm}|m{3.72cm}|m{3.72cm}|m{3.72cm}|}
\hline
\multicolumn{4}{|c|}{\textbf{Otros datos}} \\
\hline
\textbf{Frecuencia esperada} & Media (tres veces al mes) & \textbf{Rendimiento} & - \\
\hline
\textbf{Importancia} & Alta & \textbf{Urgencia} & Máxima \\
\hline
\textbf{Estado} & Implementado & \textbf{Estabilidad} & Alta \\
\hline
\end{tabular}

\vspace{1cm}

\begin{tabular}{|m{16.2cm}|}
\hline
\textbf{Comentarios} \\
\hline
- \\
\hline
\end{tabular}

\caption{Alta empleado}

\end{table}


%%%%%%%%BAJA EMPLEADO%%%%%

\begin{table}[H]
\centering
\begin{tabular}{|m{3cm}|m{4cm}|m{2cm}|m{2cm}|m{2cm}|m{1cm}|}
\hline
\textbf{Caso de uso} &  \multicolumn{4}{m{8cm}|}{Baja empleado} \vline &  \cellcolor{gray!40}CU\arabic{contadorCU}  \stepcounter{contadorCU}
\\
\hline
\textbf{Actores} & \multicolumn{5}{m{8cm}|}{Empleado y oficinista} \\
\hline
\textbf{Tipo} & \multicolumn{5}{m{8cm}|}{Real} \\
\hline
\textbf{Referencias} &\multicolumn{5}{m{8cm}|}{-} \\
\hline
\textbf{Precondición} & \multicolumn{5}{m{8cm}|}{El empleado debe estar dado de alta en el sistema.} \\
\hline
\textbf{Postcondición} & \multicolumn{5}{m{8cm}|}{El empleado ya no podrá acceder al sistema.} \\
\hline
\textbf{Autor} & Carlos Sánchez Páez & \textbf{Fecha} & 08/04/2018 & \textbf{Versión} & 2.0 \\
\hline
\end{tabular}

\vspace{1cm}

\begin{tabular}{|m{16.2cm}|}
\hline
\textbf{Propósito} \\
\hline
Eliminar los datos de un empleado. \\
\hline
\end{tabular}

\vspace{1cm}

\begin{tabular}{|m{16.2cm}|}
\hline
\textbf{Resumen} \\
\hline
Los datos del empleado son eliminados del sistema, perdiendo todo el acceso al mismo. \\
\hline
\end{tabular}

\vspace{1cm}

\begin{tabular}{|m{4pt}|m{7.33cm}|m{4pt}|m{7.33cm}|}
\hline
\multicolumn{4}{|c|}{\textbf{Curso normal}} \\
\hline
\textbf{1} & El oficinista solicita al empleado sus datos y los introduce en el sistema. & \textbf{2} & El sistema verifica los datos y solicita confirmación \\
\hline
\textbf{3} & El oficinista confirma la baja del empleado.  & \textbf{4} & El sistema elimina los datos del empleado. \\
\hline
\end{tabular}

\vspace{1cm}

\begin{tabular}{|m{10pt}|m{7.15cm}|m{10pt}|m{7.15cm}|}
\hline
\multicolumn{4}{|m{16.2cm}|}{\textbf{Cursos alternos}} \\
\hline
\multicolumn{4}{|m{16.2cm}|}{\textbf{2a) Los datos no son válidos}} \\
\hline
\multicolumn{4}{|m{16.2cm}|}{El sistema informa del error y solicita que los datos se vuelvan a introducir.} \\
\hline
\end{tabular}

\vspace{1cm}

\begin{tabular}{|m{3.72cm}|m{3.72cm}|m{3.72cm}|m{3.72cm}|}
\hline
\multicolumn{4}{|c|}{\textbf{Otros datos}} \\
\hline
\textbf{Frecuencia esperada} & Baja (una vez cada dos meses) & \textbf{Rendimiento} & - \\
\hline
\textbf{Importancia} & Alta & \textbf{Urgencia} & Máxima \\
\hline
\textbf{Estado} & En desarrollo & \textbf{Estabilidad} & Alta \\
\hline
\end{tabular}

\vspace{1cm}

\begin{tabular}{|m{16.2cm}|}
\hline
\textbf{Comentarios} \\
\hline
- \\
\hline
\end{tabular}

\caption{Baja empleado}

\end{table}

%%%%%%%%%%MODIFICAR DATOS PERSONALES EMPLEADO%%%%%%%%%%%	
\begin{table}[H]
\centering
\begin{tabular}{|m{3cm}|m{4cm}|m{2cm}|m{2cm}|m{2cm}|m{1cm}|}
\hline
\textbf{Caso de uso} &  \multicolumn{4}{m{8cm}|}{Modificar datos personales del empleado} \vline &  \cellcolor{gray!40}CU\arabic{contadorCU}  \stepcounter{contadorCU}
\\
\hline
\textbf{Actores} & \multicolumn{5}{m{8cm}|}{Empleado} \\
\hline
\textbf{Tipo} & \multicolumn{5}{m{8cm}|}{} \\
\hline
\textbf{Referencias} &\multicolumn{5}{m{8cm}|}{} \\
\hline
\textbf{Precondición} & \multicolumn{5}{m{8cm}|}{El empleado debe estar registrado en el sistema.} \\
\hline
\textbf{Postcondición} & \multicolumn{5}{m{8cm}|}{Los datos antiguos del empleado son eliminados.} \\
\hline
\textbf{Autor} & Carlos Sánchez Páez & \textbf{Fecha} & 08/04/2018 & \textbf{Versión} & 2.0 \\
\hline
\end{tabular}

\vspace{1cm}

\begin{tabular}{|m{16.2cm}|}
\hline
\textbf{Propósito} \\
\hline
Actualizar los datos de un empleado. \\
\hline
\end{tabular}

\vspace{1cm}

\begin{tabular}{|m{16.2cm}|}
\hline
\textbf{Resumen} \\
\hline
Sustituir datos antiguos de un empleado por otros más actualizados. \\
\hline
\end{tabular}

\vspace{1cm}

\begin{tabular}{|m{4pt}|m{7.33cm}|m{4pt}|m{7.33cm}|}
\hline
\multicolumn{4}{|c|}{\textbf{Curso normal}} \\
\hline
\textbf{1} & El empleado se identifica en el sistema. & \textbf{2} & El sistema valida los datos. \\
\hline
\textbf{3} & El empleado solicita modificar los datos. & \textbf{4} & El sistema muestra los datos actuales. \\
\hline
\textbf{5} & El empleado introduce los nuevos datos y confirma el cambio. & \textbf{6} & El sistema actualiza los datos. \\
\hline
\end{tabular}

\vspace{1cm}

\begin{tabular}{|m{10pt}|m{7.15cm}|m{10pt}|m{7.15cm}|}
\hline
\multicolumn{4}{|m{16.2cm}|}{\textbf{Cursos alternos}} \\
\hline
\multicolumn{4}{|m{16.2cm}|}{\textbf{2a) Los datos no son válidos}} \\
\hline
\multicolumn{4}{|m{16.2cm}|}{El sistema informa del error y solicita que los datos se vuelvan a introducir.} \\
\hline
\end{tabular}

\vspace{1cm}

\begin{tabular}{|m{3.72cm}|m{3.72cm}|m{3.72cm}|m{3.72cm}|}
\hline
\multicolumn{4}{|c|}{\textbf{Otros datos}} \\
\hline
\textbf{Frecuencia esperada} & Media (una vez al mes) & \textbf{Rendimiento} & - \\
\hline
\textbf{Importancia} & Alta & \textbf{Urgencia} & Alta \\
\hline
\textbf{Estado} & En desarrollo & \textbf{Estabilidad} & Alta \\
\hline
\end{tabular}

\vspace{1cm}

\begin{tabular}{|m{16.2cm}|}
\hline
\textbf{Comentarios} \\
\hline
- \\
\hline
\end{tabular}

\caption{Modificar datos personales del empleado}

\end{table}

%%%%%%%%%%%%CAMBIAR DISPONIBILIDAD%%%%%%%%%%%%

\begin{table}[H]
\centering
\begin{tabular}{|m{3cm}|m{4cm}|m{2cm}|m{2cm}|m{2cm}|m{1cm}|}
\hline
\textbf{Caso de uso} &  \multicolumn{4}{m{8cm}|}{Cambiar disponibilidad} \vline &  \cellcolor{gray!40}CU\arabic{contadorCU}  \stepcounter{contadorCU}
\\
\hline
\textbf{Actores} & \multicolumn{5}{m{8cm}|}{Oficinista.} \\
\hline
\textbf{Tipo} & \multicolumn{5}{m{8cm}|}{Real.} \\
\hline
\textbf{Referencias} &\multicolumn{5}{m{8cm}|}{-} \\
\hline
\textbf{Precondición} & \multicolumn{5}{m{8cm}|}{El oficinista debe estar dado de alta en el sistema.} \\
\hline
\textbf{Postcondición} & \multicolumn{5}{m{8cm}|}{-} \\
\hline
\textbf{Autor} & Carlos Sánchez Páez & \textbf{Fecha} & 08/04/2018 & \textbf{Versión} & 2.0 \\
\hline
\end{tabular}

\vspace{1cm}

\begin{tabular}{|m{16.2cm}|}
\hline
\textbf{Propósito} \\
\hline
Cambiar la disponibilidad del oficinista. \\
\hline
\end{tabular}

\vspace{1cm}

\begin{tabular}{|m{16.2cm}|}
\hline
\textbf{Resumen} \\
\hline
El oficinista actualiza sus datos de disponibilidad. \\
\hline
\end{tabular}

\vspace{1cm}

\begin{tabular}{|m{4pt}|m{7.33cm}|m{4pt}|m{7.33cm}|}
\hline
\multicolumn{4}{|c|}{\textbf{Curso normal}} \\
\hline
\textbf{1} & El oficinista se identifica en el sistema. & \textbf{2} & El sistema valida los datos. \\
\hline
\textbf{3} & El oficinista solicita modificar su disponibilidad. & \textbf{4} & El sistema muestra los datos actuales. \\
\hline
\textbf{5} & El oficinista introduce los nuevos datos y confirma el cambio. & \textbf{6} & El sistema actualiza los datos. \\
\hline
\end{tabular}

\vspace{1cm}

\begin{tabular}{|m{10pt}|m{7.15cm}|m{10pt}|m{7.15cm}|}
\hline
\multicolumn{4}{|m{16.2cm}|}{\textbf{Cursos alternos}} \\
\hline
\multicolumn{4}{|m{16.2cm}|}{\textbf{2a) Los datos no son válidos}} \\
\hline
\multicolumn{4}{|m{16.2cm}|}{El sistema informa del error y solicita que los datos se vuelvan a introducir.} \\
\hline
\end{tabular}

\vspace{1cm}

\begin{tabular}{|m{3.72cm}|m{3.72cm}|m{3.72cm}|m{3.72cm}|}
\hline
\multicolumn{4}{|c|}{\textbf{Otros datos}} \\
\hline
\textbf{Frecuencia esperada} & Media (una vez al mes) & \textbf{Rendimiento} & - \\
\hline
\textbf{Importancia} & Alta & \textbf{Urgencia} & Alta \\
\hline
\textbf{Estado} & En desarrollo & \textbf{Estabilidad} & Alta \\
\hline
\end{tabular}

\vspace{1cm}

\begin{tabular}{|m{16.2cm}|}
\hline
\textbf{Comentarios} \\
\hline
- \\
\hline
\end{tabular}

\caption{Cambiar disponibilidad}

\end{table}

%%%%%%%%%%%%%%ASIGNAR RUTA DE TRANSPORTE%%%%%%%%%%

\begin{table}[H]
\centering
\begin{tabular}{|m{3cm}|m{4cm}|m{2cm}|m{2cm}|m{2cm}|m{1cm}|}
\hline
\textbf{Caso de uso} &  \multicolumn{4}{m{8cm}|}{Asignar ruta de transporte} \vline &  \cellcolor{gray!40}CU\arabic{contadorCU}  \stepcounter{contadorCU}
\\
\hline
\textbf{Actores} & \multicolumn{5}{m{8cm}|}{Conductor y oficinista.} \\
\hline
\textbf{Tipo} & \multicolumn{5}{m{8cm}|}{Real.} \\
\hline
\textbf{Referencias} &\multicolumn{5}{m{8cm}|}{-} \\
\hline
\textbf{Precondición} & \multicolumn{5}{m{8cm}|}{Los actores deben estar dados de alta en el sistema.} \\
\hline
\textbf{Postcondición} & \multicolumn{5}{m{8cm}|}{-} \\
\hline
\textbf{Autor} & Carlos Sánchez Páez & \textbf{Fecha} & 08/04/2018 & \textbf{Versión} & 2.0 \\
\hline
\end{tabular}

\vspace{1cm}

\begin{tabular}{|m{16.2cm}|}
\hline
\textbf{Propósito} \\
\hline
Establecer destinos a visitar por el conductor. \\
\hline
\end{tabular}

\vspace{1cm}

\begin{tabular}{|m{16.2cm}|}
\hline
\textbf{Resumen} \\
\hline
Se le asigna al conductor una serie de destinos que tendrá que visitar en su próxima jornada laboral. \\
\hline
\end{tabular}

\vspace{1cm}

\begin{tabular}{|m{4pt}|m{7.33cm}|m{4pt}|m{7.33cm}|}
\hline
\multicolumn{4}{|c|}{\textbf{Curso normal}} \\
\hline
\textbf{1} & El oficinista introduce los datos del conductor en el sistema. & \textbf{2} & El sistema valida los datos. \\
\hline
\textbf{3} & El oficinista solicita la ruta de transporte & \textbf{4} & El sistema muestra la ruta. \\
\hline
\textbf{5} & El oficinista informa de la ruta al conductor. & \textbf{6} &  \\
\hline
\end{tabular}

\vspace{1cm}

\begin{tabular}{|m{10pt}|m{7.15cm}|m{10pt}|m{7.15cm}|}
\hline
\multicolumn{4}{|m{16.2cm}|}{\textbf{Cursos alternos}} \\
\hline
\multicolumn{4}{|m{16.2cm}|}{\textbf{2a) Los datos no son válidos}} \\
\hline
\multicolumn{4}{|m{16.2cm}|}{El sistema informa del error y solicita que los datos se vuelvan a introducir.} \\
\hline
\multicolumn{4}{|m{16.2cm}|}{\textbf{3a) No se han introducido los destinos a visitar}} \\
\hline
\multicolumn{4}{|m{16.2cm}|}{El sistema informa del error y solicita que los datos se introduzcan.} \\
\hline
\end{tabular}

\vspace{1cm}

\begin{tabular}{|m{3.72cm}|m{3.72cm}|m{3.72cm}|m{3.72cm}|}
\hline
\multicolumn{4}{|c|}{\textbf{Otros datos}} \\
\hline
\textbf{Frecuencia esperada} & Alta (a diario) & \textbf{Rendimiento} & - \\
\hline
\textbf{Importancia} & Extrema & \textbf{Urgencia} & Máxima \\
\hline
\textbf{Estado} & Implementado & \textbf{Estabilidad} & Alta \\
\hline
\end{tabular}

\vspace{1cm}

\begin{tabular}{|m{16.2cm}|}
\hline
\textbf{Comentarios} \\
\hline
- \\
\hline
\end{tabular}

\caption{Asignar ruta de transporte}

\end{table}


%%%%%%%%CONSULTAR ESTADO DE CONDUCTOR%%%%%%%%%%%

\begin{table}[H]
\centering
\begin{tabular}{|m{3cm}|m{4cm}|m{2cm}|m{2cm}|m{2cm}|m{1cm}|}
\hline
\textbf{Caso de uso} &  \multicolumn{4}{m{8cm}|}{Consultar estado de conductor} \vline &  \cellcolor{gray!40}CU\arabic{contadorCU}  \stepcounter{contadorCU}
\\
\hline
\textbf{Actores} & \multicolumn{5}{m{8cm}|}{Oficinista} \\
\hline
\textbf{Tipo} & \multicolumn{5}{m{8cm}|}{Real} \\
\hline
\textbf{Referencias} &\multicolumn{5}{m{8cm}|}{-} \\
\hline
\textbf{Precondición} & \multicolumn{5}{m{8cm}|}{El conductor debe estar dado de alta en el sistema.} \\
\hline
\textbf{Postcondición} & \multicolumn{5}{m{8cm}|}{-} \\
\hline
\textbf{Autor} & Carlos Sánchez Páez & \textbf{Fecha} & 08/04/2018 & \textbf{Versión} & 2.0 \\
\hline
\end{tabular}

\vspace{1cm}

\begin{tabular}{|m{16.2cm}|}
\hline
\textbf{Propósito} \\
\hline
Ver información relativa al conductor. \\
\hline
\end{tabular}

\vspace{1cm}

\begin{tabular}{|m{16.2cm}|}
\hline
\textbf{Resumen} \\
\hline
El oficinista podrá obtener información sobre el conductor (destinos realizados/pendientes, furgoneta asociada, etc.) \\
\hline
\end{tabular}

\vspace{1cm}

\begin{tabular}{|m{4pt}|m{7.33cm}|m{4pt}|m{7.33cm}|}
\hline
\multicolumn{4}{|c|}{\textbf{Curso normal}} \\
\hline
\textbf{1} & El oficinista introduce los datos del conductor en el sistema. & \textbf{2} & El sistema valida los datos. \\
\hline
\textbf{3} & El oficinista solicita la información requerida. & \textbf{4} & El sistema muestra la información. \\
\hline
\end{tabular}

\vspace{1cm}

\begin{tabular}{|m{10pt}|m{7.15cm}|m{10pt}|m{7.15cm}|}
\hline
\multicolumn{4}{|m{16.2cm}|}{\textbf{Cursos alternos}} \\
\hline
\multicolumn{4}{|m{16.2cm}|}{\textbf{2a) Los datos no son válidos}} \\
\hline
\multicolumn{4}{|m{16.2cm}|}{El sistema informa del error y solicita que los datos se vuelvan a introducir.} \\
\hline
\end{tabular}

\vspace{1cm}

\begin{tabular}{|m{3.72cm}|m{3.72cm}|m{3.72cm}|m{3.72cm}|}
\hline
\multicolumn{4}{|c|}{\textbf{Otros datos}} \\
\hline
\textbf{Frecuencia esperada} & Alta (a diario) & \textbf{Rendimiento} & - \\
\hline
\textbf{Importancia} & Extrema & \textbf{Urgencia} & Máxima \\
\hline
\textbf{Estado} & En desarrollo & \textbf{Estabilidad} & Alta \\
\hline
\end{tabular}

\vspace{1cm}

\begin{tabular}{|m{16.2cm}|}
\hline
\textbf{Comentarios} \\
\hline
- \\
\hline
\end{tabular}

\caption{Consultar estado de conductor}

\end{table}

%%%%%%%%%%ALTA SUCURSAL%%%%%%%%%%

\begin{table}[H]
\centering
\begin{tabular}{|m{3cm}|m{4cm}|m{2cm}|m{2cm}|m{2cm}|m{1cm}|}
\hline
\textbf{Caso de uso} &  \multicolumn{4}{m{8cm}|}{Alta sucursal} \vline &  \cellcolor{gray!40}CU\arabic{contadorCU}  \stepcounter{contadorCU}
\\
\hline
\textbf{Actores} & \multicolumn{5}{m{8cm}|}{Oficinista} \\
\hline
\textbf{Tipo} & \multicolumn{5}{m{8cm}|}{Real} \\
\hline
\textbf{Referencias} &\multicolumn{5}{m{8cm}|}{CU8,CU9} \\
\hline
\textbf{Precondición} & \multicolumn{5}{m{8cm}|}{-} \\
\hline
\textbf{Postcondición} & \multicolumn{5}{m{8cm}|}{-} \\
\hline
\textbf{Autor} & Carlos Sánchez Páez & \textbf{Fecha} & 08/04/2018 & \textbf{Versión} & 2.0 \\
\hline
\end{tabular}

\vspace{1cm}

\begin{tabular}{|m{16.2cm}|}
\hline
\textbf{Propósito} \\
\hline
Incorporar una nueva sucursal al sistema. \\
\hline
\end{tabular}

\vspace{1cm}

\begin{tabular}{|m{16.2cm}|}
\hline
\textbf{Resumen} \\
\hline
Una nueva sucursal pasará a formar parte de la empresa, pudiendo acceder sus empleados al sistema. \\
\hline
\end{tabular}

\vspace{1cm}

\begin{tabular}{|m{4pt}|m{7.33cm}|m{4pt}|m{7.33cm}|}
\hline
\multicolumn{4}{|c|}{\textbf{Curso normal}} \\
\hline
\textbf{1} & El oficinista inicia el proceso de alta. & \textbf{2} & El sistema ofrece un formulario a rellenar. \\
\hline
\textbf{3} & El oficinista rellena el formulario. & \textbf{4} & El sistema comprueba los datos. \\
\hline
\textbf{5} & & \textbf{6} & incluir(CU8, alta oficina) \\
\hline
\end{tabular}

\vspace{1cm}

\begin{tabular}{|m{10pt}|m{7.15cm}|m{10pt}|m{7.15cm}|}
\hline
\multicolumn{4}{|m{16.2cm}|}{\textbf{Cursos alternos}} \\
\hline
\multicolumn{4}{|m{16.2cm}|}{\textbf{4a) Los datos no son válidos}} \\
\hline
\multicolumn{4}{|m{16.2cm}|}{El sistema informa del error y solicita que los datos se vuelvan a introducir.} \\
\hline
\end{tabular}

\vspace{1cm}

\begin{tabular}{|m{3.72cm}|m{3.72cm}|m{3.72cm}|m{3.72cm}|}
\hline
\multicolumn{4}{|c|}{\textbf{Otros datos}} \\
\hline
\textbf{Frecuencia esperada} & Media (una vez al trimestre) & \textbf{Rendimiento} & - \\
\hline
\textbf{Importancia} & Alta & \textbf{Urgencia} & Media \\
\hline
\textbf{Estado} & En desarrollo & \textbf{Estabilidad} & Alta \\
\hline
\end{tabular}

\vspace{1cm}

\begin{tabular}{|m{16.2cm}|}
\hline
\textbf{Comentarios} \\
\hline
- \\
\hline
\end{tabular}

\caption{Alta sucursal}

\end{table}


%%%%%%%%%ALTA OFICINA%%%%%%

\begin{table}[H]
\centering
\begin{tabular}{|m{3cm}|m{4cm}|m{2cm}|m{2cm}|m{2cm}|m{1cm}|}
\hline
\textbf{Caso de uso} &  \multicolumn{4}{m{8cm}|}{Alta oficina} \vline &  \cellcolor{gray!40}CU\arabic{contadorCU}  \stepcounter{contadorCU}
\\
\hline
\textbf{Actores} & \multicolumn{5}{m{8cm}|}{Oficinista} \\
\hline
\textbf{Tipo} & \multicolumn{5}{m{8cm}|}{Real} \\
\hline
\textbf{Referencias} &\multicolumn{5}{m{8cm}|}{-} \\
\hline
\textbf{Precondición} & \multicolumn{5}{m{8cm}|}{-} \\
\hline
\textbf{Postcondición} & \multicolumn{5}{m{8cm}|}{-} \\
\hline
\textbf{Autor} & Carlos Sánchez Páez & \textbf{Fecha} & 08/04/2018 & \textbf{Versión} & 2.0 \\
\hline
\end{tabular}

\vspace{1cm}

\begin{tabular}{|m{16.2cm}|}
\hline
\textbf{Propósito} \\
\hline
Dar de alta una oficina en el sistema. \\
\hline
\end{tabular}

\vspace{1cm}

\begin{tabular}{|m{16.2cm}|}
\hline
\textbf{Resumen} \\
\hline
La oficina pasa a formar parte de la empresa. \\
\hline
\end{tabular}

\vspace{1cm}

\begin{tabular}{|m{4pt}|m{7.33cm}|m{4pt}|m{7.33cm}|}
\hline
\multicolumn{4}{|c|}{\textbf{Curso normal}} \\
\hline
\textbf{1} & El oficinista inicia el proceso de alta. & \textbf{2} & El sistema ofrece un formulario a rellenar. \\
\hline
\textbf{3} & El oficinista rellena el formulario. & \textbf{4} & El sistema comprueba los datos. \\
\hline
\textbf{5} & & \textbf{6} & El sistema almacena los datos. \\
\hline
\end{tabular}

\vspace{1cm}

\begin{tabular}{|m{10pt}|m{7.15cm}|m{10pt}|m{7.15cm}|}
\hline
\multicolumn{4}{|m{16.2cm}|}{\textbf{Cursos alternos}} \\
\hline
\multicolumn{4}{|m{16.2cm}|}{\textbf{4a) Los datos no son válidos}} \\
\hline
\multicolumn{4}{|m{16.2cm}|}{El sistema informa del error y solicita que los datos se vuelvan a introducir.} \\
\hline
\end{tabular}

\vspace{1cm}

\begin{tabular}{|m{3.72cm}|m{3.72cm}|m{3.72cm}|m{3.72cm}|}
\hline
\multicolumn{4}{|c|}{\textbf{Otros datos}} \\
\hline
\textbf{Frecuencia esperada} & Media (una vez al trimestre) & \textbf{Rendimiento} & - \\
\hline
\textbf{Importancia} & Alta & \textbf{Urgencia} & Media \\
\hline
\textbf{Estado} & Implementado & \textbf{Estabilidad} & Alta \\
\hline
\end{tabular}

\vspace{1cm}

\begin{tabular}{|m{16.2cm}|}
\hline
\textbf{Comentarios} \\
\hline
- \\
\hline
\end{tabular}

\caption{Alta oficina}

\end{table}

%%%%%%%%%ALTA ALMACÉN%%%%%

\begin{table}[H]
\centering
\begin{tabular}{|m{3cm}|m{4cm}|m{2cm}|m{2cm}|m{2cm}|m{1cm}|}
\hline
\textbf{Caso de uso} &  \multicolumn{4}{m{8cm}|}{Alta almacén} \vline &  \cellcolor{gray!40}CU\arabic{contadorCU}  \stepcounter{contadorCU}
\\
\hline
\textbf{Actores} & \multicolumn{5}{m{8cm}|}{Oficinista} \\
\hline
\textbf{Tipo} & \multicolumn{5}{m{8cm}|}{Real} \\
\hline
\textbf{Referencias} &\multicolumn{5}{m{8cm}|}{-} \\
\hline
\textbf{Precondición} & \multicolumn{5}{m{8cm}|}{-} \\
\hline
\textbf{Postcondición} & \multicolumn{5}{m{8cm}|}{-} \\
\hline
\textbf{Autor} & Carlos Sánchez Páez & \textbf{Fecha} & 08/04/2018 & \textbf{Versión} & 2.0 \\
\hline
\end{tabular}

\vspace{1cm}

\begin{tabular}{|m{16.2cm}|}
\hline
\textbf{Propósito} \\
\hline
Dar de alta un almacén en el sistema. \\
\hline
\end{tabular}

\vspace{1cm}

\begin{tabular}{|m{16.2cm}|}
\hline
\textbf{Resumen} \\
\hline
El almacén pasa a formar parte de la empresa, siendo tomado en cuenta para la planificación de rutas. \\
\hline
\end{tabular}

\vspace{1cm}

\begin{tabular}{|m{4pt}|m{7.33cm}|m{4pt}|m{7.33cm}|}
\hline
\multicolumn{4}{|c|}{\textbf{Curso normal}} \\
\hline
\textbf{1} & El oficinista inicia el proceso de alta. & \textbf{2} & El sistema ofrece un formulario a rellenar. \\
\hline
\textbf{3} & El oficinista rellena el formulario. & \textbf{4} & El sistema comprueba los datos. \\
\hline
\textbf{5} & & \textbf{6} & El sistema almacena los datos. \\
\hline
\end{tabular}

\vspace{1cm}

\begin{tabular}{|m{10pt}|m{7.15cm}|m{10pt}|m{7.15cm}|}
\hline
\multicolumn{4}{|m{16.2cm}|}{\textbf{Cursos alternos}} \\
\hline
\multicolumn{4}{|m{16.2cm}|}{\textbf{4a) Los datos no son válidos}} \\
\hline
\multicolumn{4}{|m{16.2cm}|}{El sistema informa del error y solicita que los datos se vuelvan a introducir.} \\
\hline
\end{tabular}

\vspace{1cm}

\begin{tabular}{|m{3.72cm}|m{3.72cm}|m{3.72cm}|m{3.72cm}|}
\hline
\multicolumn{4}{|c|}{\textbf{Otros datos}} \\
\hline
\textbf{Frecuencia esperada} & Media (una vez al trimestre) & \textbf{Rendimiento} & - \\
\hline
\textbf{Importancia} & Alta & \textbf{Urgencia} & Media \\
\hline
\textbf{Estado} & En desarrollo & \textbf{Estabilidad} & Alta \\
\hline
\end{tabular}

\vspace{1cm}

\begin{tabular}{|m{16.2cm}|}
\hline
\textbf{Comentarios} \\
\hline
- \\
\hline
\end{tabular}

\caption{Alta almacén}

\end{table}


%%%%%%%%BAJA SUCURSAL%%%%%%%

\begin{table}[H]
\centering
\begin{tabular}{|m{3cm}|m{4cm}|m{2cm}|m{2cm}|m{2cm}|m{1cm}|}
\hline
\textbf{Caso de uso} &  \multicolumn{4}{m{8cm}|}{Baja sucursal} \vline &  \cellcolor{gray!40}CU\arabic{contadorCU}  \stepcounter{contadorCU}
\\
\hline
\textbf{Actores} & \multicolumn{5}{m{8cm}|}{Oficinista} \\
\hline
\textbf{Tipo} & \multicolumn{5}{m{8cm}|}{Real} \\
\hline
\textbf{Referencias} &\multicolumn{5}{m{8cm}|}{-} \\
\hline
\textbf{Precondición} & \multicolumn{5}{m{8cm}|}{La sucursal debe estar dada de alta en el sistema.} \\
\hline
\textbf{Postcondición} & \multicolumn{5}{m{8cm}|}{-} \\
\hline
\textbf{Autor} & Carlos Sánchez Páez & \textbf{Fecha} & 08/04/2018 & \textbf{Versión} & 2.0 \\
\hline
\end{tabular}

\vspace{1cm}

\begin{tabular}{|m{16.2cm}|}
\hline
\textbf{Propósito} \\
\hline
Eliminar una sucursal del sistema. \\
\hline
\end{tabular}

\vspace{1cm}

\begin{tabular}{|m{16.2cm}|}
\hline
\textbf{Resumen} \\
\hline
La sucursal deja de formar parte de la empresa y se elimina del sistema. \\
\hline
\end{tabular}

\vspace{1cm}

\begin{tabular}{|m{4pt}|m{7.33cm}|m{4pt}|m{7.33cm}|}
\hline
\multicolumn{4}{|c|}{\textbf{Curso normal}} \\
\hline
\textbf{1} & El oficinista inicia el proceso de baja. & \textbf{2} & El sistema solicita los datos de la sucursal. \\
\hline
\textbf{3} & El oficinista rellena el formulario. & \textbf{4} & El sistema comprueba los datos. \\
\hline
\textbf{5} & & \textbf{6} & El sistema solicita confirmación. \\
\hline
\textbf{7} & El oficinista confirma la baja. & \textbf{8} & El sistema elimina los datos.  \\
\hline
\end{tabular}

\vspace{1cm}

\begin{tabular}{|m{10pt}|m{7.15cm}|m{10pt}|m{7.15cm}|}
\hline
\multicolumn{4}{|m{16.2cm}|}{\textbf{Cursos alternos}} \\
\hline
\multicolumn{4}{|m{16.2cm}|}{\textbf{4a) Los datos no son válidos}} \\
\hline
\multicolumn{4}{|m{16.2cm}|}{El sistema informa del error y solicita que los datos se vuelvan a introducir.} \\
\hline
\end{tabular}

\vspace{1cm}

\begin{tabular}{|m{3.72cm}|m{3.72cm}|m{3.72cm}|m{3.72cm}|}
\hline
\multicolumn{4}{|c|}{\textbf{Otros datos}} \\
\hline
\textbf{Frecuencia esperada} & Baja (una vez al año) & \textbf{Rendimiento} & - \\
\hline
\textbf{Importancia} & Media & \textbf{Urgencia} & Media \\
\hline
\textbf{Estado} & En desarrollo & \textbf{Estabilidad} & Alta \\
\hline
\end{tabular}

\vspace{1cm}

\begin{tabular}{|m{16.2cm}|}
\hline
\textbf{Comentarios} \\
\hline
- \\
\hline
\end{tabular}

\caption{Baja sucursal}

\end{table}

%%%%%%%MODIFICACIÓN SUCURSAL%%%%%

\begin{table}[H]
\centering
\begin{tabular}{|m{3cm}|m{4cm}|m{2cm}|m{2cm}|m{2cm}|m{1cm}|}
\hline
\textbf{Caso de uso} &  \multicolumn{4}{m{8cm}|}{Modificación sucursal} \vline &  \cellcolor{gray!40}CU\arabic{contadorCU}  \stepcounter{contadorCU}
\\
\hline
\textbf{Actores} & \multicolumn{5}{m{8cm}|}{Oficinista} \\
\hline
\textbf{Tipo} & \multicolumn{5}{m{8cm}|}{Real} \\
\hline
\textbf{Referencias} &\multicolumn{5}{m{8cm}|}{-} \\
\hline
\textbf{Precondición} & \multicolumn{5}{m{8cm}|}{La sucursal debe estar dada de alta.} \\
\hline
\textbf{Postcondición} & \multicolumn{5}{m{8cm}|}{Los datos antiguos se eliminan} \\
\hline
\textbf{Autor} & Carlos Sánchez Páez & \textbf{Fecha} & 08/04/2018 & \textbf{Versión} & 2.0 \\
\hline
\end{tabular}

\vspace{1cm}

\begin{tabular}{|m{16.2cm}|}
\hline
\textbf{Propósito} \\
\hline
Actualizar los datos de una sucursal. \\
\hline
\end{tabular}

\vspace{1cm}

\begin{tabular}{|m{16.2cm}|}
\hline
\textbf{Resumen} \\
\hline
Los datos de una sucursal son sustituidos por otros nuevos en el sistema. \\
\hline
\end{tabular}

\vspace{1cm}

\begin{tabular}{|m{4pt}|m{7.33cm}|m{4pt}|m{7.33cm}|}
\hline
\multicolumn{4}{|c|}{\textbf{Curso normal}} \\
\hline
\textbf{1} & El oficinista inicia el proceso de modificación. & \textbf{2} & El sistema solicita los datos de la sucursal. \\
\hline
\textbf{3} & El oficinista introduce los datos. & \textbf{4} & El sistema comprueba los datos. \\
\hline
\textbf{5} & El oficinista introduce los nuevos datos. & \textbf{6} & El sistema solicita confirmación. \\
\hline
\textbf{7} & El oficinista confirma los cambios. & \textbf{8} & El sistema almacena los nuevos datos. \\
\hline
\end{tabular}

\vspace{1cm}

\begin{tabular}{|m{10pt}|m{7.15cm}|m{10pt}|m{7.15cm}|}
\hline
\multicolumn{4}{|m{16.2cm}|}{\textbf{Cursos alternos}} \\
\hline
\multicolumn{4}{|m{16.2cm}|}{\textbf{4a) Los datos no son válidos}} \\
\hline
\multicolumn{4}{|m{16.2cm}|}{El sistema informa del error y solicita que los datos se vuelvan a introducir.} \\
\hline
\end{tabular}

\vspace{1cm}

\begin{tabular}{|m{3.72cm}|m{3.72cm}|m{3.72cm}|m{3.72cm}|}
\hline
\multicolumn{4}{|c|}{\textbf{Otros datos}} \\
\hline
\textbf{Frecuencia esperada} & Media (una vez al trimestre) & \textbf{Rendimiento} & - \\
\hline
\textbf{Importancia} & Media & \textbf{Urgencia} & Media \\
\hline
\textbf{Estado} & En desarrollo & \textbf{Estabilidad} & Alta \\
\hline
\end{tabular}

\vspace{1cm}

\begin{tabular}{|m{16.2cm}|}
\hline
\textbf{Comentarios} \\
\hline
- \\
\hline
\end{tabular}

\caption{Modificación sucursal}

\end{table}

%%%%%%CONSULTA SUCURSAL%%%%%%%

\begin{table}[H]
\centering
\begin{tabular}{|m{3cm}|m{4cm}|m{2cm}|m{2cm}|m{2cm}|m{1cm}|}
\hline
\textbf{Caso de uso} &  \multicolumn{4}{m{8cm}|}{Consulta sucursal} \vline &  \cellcolor{gray!40}CU\arabic{contadorCU}  \stepcounter{contadorCU}
\\
\hline
\textbf{Actores} & \multicolumn{5}{m{8cm}|}{Empleado} \\
\hline
\textbf{Tipo} & \multicolumn{5}{m{8cm}|}{Real} \\
\hline
\textbf{Referencias} &\multicolumn{5}{m{8cm}|}{-} \\
\hline
\textbf{Precondición} & \multicolumn{5}{m{8cm}|}{La sucursal debe estar dada de alta.} \\
\hline
\textbf{Postcondición} & \multicolumn{5}{m{8cm}|}{-} \\
\hline
\textbf{Autor} & Carlos Sánchez Páez & \textbf{Fecha} & 08/04/2018 & \textbf{Versión} & 2.0 \\
\hline
\end{tabular}

\vspace{1cm}

\begin{tabular}{|m{16.2cm}|}
\hline
\textbf{Propósito} \\
\hline
Ver datos relacionados con una sucursal. \\
\hline
\end{tabular}

\vspace{1cm}

\begin{tabular}{|m{16.2cm}|}
\hline
\textbf{Resumen} \\
\hline
El empleado podrá obtener información relativa a una determinada sucursal del sistema. \\
\hline
\end{tabular}

\vspace{1cm}

\begin{tabular}{|m{4pt}|m{7.33cm}|m{4pt}|m{7.33cm}|}
\hline
\multicolumn{4}{|c|}{\textbf{Curso normal}} \\
\hline
\textbf{1} & El empleado inicia el proceso de consulta. & \textbf{2} & El sistema solicita los datos de la sucursal. \\
\hline
\textbf{3} & El empleado introduce los datos. & \textbf{4} & El sistema comprueba los datos. \\
\hline
\textbf{5} & & \textbf{6} & El sistema muestra los datos solicitados. \\
\hline

\end{tabular}

\vspace{1cm}

\begin{tabular}{|m{10pt}|m{7.15cm}|m{10pt}|m{7.15cm}|}
\hline
\multicolumn{4}{|m{16.2cm}|}{\textbf{Cursos alternos}} \\
\hline
\multicolumn{4}{|m{16.2cm}|}{\textbf{4a) Los datos no son válidos}} \\
\hline
\multicolumn{4}{|m{16.2cm}|}{El sistema informa del error y solicita que los datos se vuelvan a introducir.} \\
\hline
\end{tabular}

\vspace{1cm}

\begin{tabular}{|m{3.72cm}|m{3.72cm}|m{3.72cm}|m{3.72cm}|}
\hline
\multicolumn{4}{|c|}{\textbf{Otros datos}} \\
\hline
\textbf{Frecuencia esperada} & Alta (a diario) & \textbf{Rendimiento} & - \\
\hline
\textbf{Importancia} & Extrema & \textbf{Urgencia} & Máxima \\
\hline
\textbf{Estado} & En desarrollo & \textbf{Estabilidad} & Alta \\
\hline
\end{tabular}

\vspace{1cm}

\begin{tabular}{|m{16.2cm}|}
\hline
\textbf{Comentarios} \\
\hline
- \\
\hline
\end{tabular}

\caption{Consulta sucursal}

\end{table}

%%%%%ALTA FURGONETA%%%%

\begin{table}[H]
\centering
\begin{tabular}{|m{3cm}|m{4cm}|m{2cm}|m{2cm}|m{2cm}|m{1cm}|}
\hline
\textbf{Caso de uso} &  \multicolumn{4}{m{8cm}|}{Alta furgoneta} \vline &  \cellcolor{gray!40}CU\arabic{contadorCU}  \stepcounter{contadorCU}
\\
\hline
\textbf{Actores} & \multicolumn{5}{m{8cm}|}{Oficinista} \\
\hline
\textbf{Tipo} & \multicolumn{5}{m{8cm}|}{Real} \\
\hline
\textbf{Referencias} &\multicolumn{5}{m{8cm}|}{-} \\
\hline
\textbf{Precondición} & \multicolumn{5}{m{8cm}|}{-} \\
\hline
\textbf{Postcondición} & \multicolumn{5}{m{8cm}|}{-} \\
\hline
\textbf{Autor} & Carlos Sánchez Páez & \textbf{Fecha} & 08/04/2018 & \textbf{Versión} & 2.0 \\
\hline
\end{tabular}

\vspace{1cm}

\begin{tabular}{|m{16.2cm}|}
\hline
\textbf{Propósito} \\
\hline
Dar de alta una nueva furgoneta. \\
\hline
\end{tabular}

\vspace{1cm}

\begin{tabular}{|m{16.2cm}|}
\hline
\textbf{Resumen} \\
\hline
Hacer que una furgoneta pase a formar parte del sistema, teniéndola en cuenta para la planificación de rutas y otros trámites. \\
\hline
\end{tabular}

\vspace{1cm}


\begin{tabular}{|m{4pt}|m{7.33cm}|m{4pt}|m{7.33cm}|}
\hline
\multicolumn{4}{|c|}{\textbf{Curso normal}} \\
\hline
\textbf{1} & El oficinista inicia el proceso de alta. & \textbf{2} & El sistema solicita los datos de la furgoneta. \\
\hline
\textbf{3} & El oficinista introduce los datos. & \textbf{4} & El sistema comprueba los datos. \\
\hline
\textbf{5} & & \textbf{6} & El sistema solicita confirmación \\
\hline
\textbf{7} & El oficinista confirma los datos. & \textbf{8} & El sistema almacena los datos. \\
\hline
\end{tabular}

\vspace{1cm}

\begin{tabular}{|m{10pt}|m{7.15cm}|m{10pt}|m{7.15cm}|}
\hline
\multicolumn{4}{|m{16.2cm}|}{\textbf{Cursos alternos}} \\
\hline
\multicolumn{4}{|m{16.2cm}|}{\textbf{4a) Los datos no son válidos}} \\
\hline
\multicolumn{4}{|m{16.2cm}|}{El sistema informa del error y solicita que los datos se vuelvan a introducir.} \\
\hline
\end{tabular}

\vspace{1cm}

\begin{tabular}{|m{3.72cm}|m{3.72cm}|m{3.72cm}|m{3.72cm}|}
\hline
\multicolumn{4}{|c|}{\textbf{Otros datos}} \\
\hline
\textbf{Frecuencia esperada} & Media (dos veces al trimestre) & \textbf{Rendimiento} & - \\
\hline
\textbf{Importancia} & Alta & \textbf{Urgencia} & Máxima \\
\hline
\textbf{Estado} & En desarrollo & \textbf{Estabilidad} & Alta \\
\hline
\end{tabular}

\vspace{1cm}

\begin{tabular}{|m{16.2cm}|}
\hline
\textbf{Comentarios} \\
\hline
- \\
\hline
\end{tabular}

\caption{Alta furgoneta}

\end{table}

%%%%%BAJA FURGONETA%%%%%

\begin{table}[H]
\centering
\begin{tabular}{|m{3cm}|m{4cm}|m{2cm}|m{2cm}|m{2cm}|m{1cm}|}
\hline
\textbf{Caso de uso} &  \multicolumn{4}{m{8cm}|}{Baja furgoneta} \vline &  \cellcolor{gray!40}CU\arabic{contadorCU}  \stepcounter{contadorCU}
\\
\hline
\textbf{Actores} & \multicolumn{5}{m{8cm}|}{Oficinista} \\
\hline
\textbf{Tipo} & \multicolumn{5}{m{8cm}|}{Real} \\
\hline
\textbf{Referencias} &\multicolumn{5}{m{8cm}|}{-} \\
\hline
\textbf{Precondición} & \multicolumn{5}{m{8cm}|}{La furgoneta debe estar dada de alta en el sistema.} \\
\hline
\textbf{Postcondición} & \multicolumn{5}{m{8cm}|}{-} \\
\hline
\textbf{Autor} & Carlos Sánchez Páez & \textbf{Fecha} & 08/04/2018 & \textbf{Versión} & 2.0 \\
\hline
\end{tabular}

\vspace{1cm}

\begin{tabular}{|m{16.2cm}|}
\hline
\textbf{Propósito} \\
\hline
Eliminar una furgoneta del sistema. \\
\hline
\end{tabular}

\vspace{1cm}

\begin{tabular}{|m{16.2cm}|}
\hline
\textbf{Resumen} \\
\hline
La furgoneta deja de formar parte del sistema, por lo que no se la tiene en cuenta para el cálculo de la ruta. \\
\hline
\end{tabular}

\vspace{1cm}


\begin{tabular}{|m{4pt}|m{7.33cm}|m{4pt}|m{7.33cm}|}
\hline
\multicolumn{4}{|c|}{\textbf{Curso normal}} \\
\hline
\textbf{1} & El oficinista inicia el proceso de baja. & \textbf{2} & El sistema solicita los datos de la furgoneta. \\
\hline
\textbf{3} & El oficinista introduce los datos. & \textbf{4} & El sistema comprueba los datos. \\
\hline
\textbf{5} & & \textbf{6} & El sistema solicita confirmación \\
\hline
\textbf{7} & El oficinista confirma los datos. & \textbf{8} & El sistema elimina los datos.\\
\hline
\end{tabular}

\vspace{1cm}

\begin{tabular}{|m{10pt}|m{7.15cm}|m{10pt}|m{7.15cm}|}
\hline
\multicolumn{4}{|m{16.2cm}|}{\textbf{Cursos alternos}} \\
\hline
\multicolumn{4}{|m{16.2cm}|}{\textbf{4a) Los datos no son válidos}} \\
\hline
\multicolumn{4}{|m{16.2cm}|}{El sistema informa del error y solicita que los datos se vuelvan a introducir.} \\
\hline
\end{tabular}

\vspace{1cm}

\begin{tabular}{|m{3.72cm}|m{3.72cm}|m{3.72cm}|m{3.72cm}|}
\hline
\multicolumn{4}{|c|}{\textbf{Otros datos}} \\
\hline
\textbf{Frecuencia esperada} & Baja (una vez al año). & \textbf{Rendimiento} & - \\
\hline
\textbf{Importancia} & Media. & \textbf{Urgencia} & Media. \\
\hline
\textbf{Estado} & En desarrollo & \textbf{Estabilidad} & Alta \\
\hline
\end{tabular}

\vspace{1cm}

\begin{tabular}{|m{16.2cm}|}
\hline
\textbf{Comentarios} \\
\hline
- \\
\hline
\end{tabular}

\caption{Baja furgoneta}

\end{table}

%%%%%%%%MODIFICACIÓN FURGONETA%%%%%

\begin{table}[H]
\centering
\begin{tabular}{|m{3cm}|m{4cm}|m{2cm}|m{2cm}|m{2cm}|m{1cm}|}
\hline
\textbf{Caso de uso} &  \multicolumn{4}{m{8cm}|}{Modificación furgoneta} \vline &  \cellcolor{gray!40}CU\arabic{contadorCU}  \stepcounter{contadorCU}
\\
\hline
\textbf{Actores} & \multicolumn{5}{m{8cm}|}{Oficinista} \\
\hline
\textbf{Tipo} & \multicolumn{5}{m{8cm}|}{Real} \\
\hline
\textbf{Referencias} &\multicolumn{5}{m{8cm}|}{-} \\
\hline
\textbf{Precondición} & \multicolumn{5}{m{8cm}|}{La furgoneta debe estar dada de alta en el sistema} \\
\hline
\textbf{Postcondición} & \multicolumn{5}{m{8cm}|}{-} \\
\hline
\textbf{Autor} & Carlos Sánchez Páez & \textbf{Fecha} & 08/04/2018 & \textbf{Versión} & 2.0 \\
\hline
\end{tabular}

\vspace{1cm}

\begin{tabular}{|m{16.2cm}|}
\hline
\textbf{Propósito} \\
\hline
Actualizar datos de una furgoneta. \\
\hline
\end{tabular}

\vspace{1cm}

\begin{tabular}{|m{16.2cm}|}
\hline
\textbf{Resumen} \\
\hline
Los datos de una determinada furgoneta serán actualizados para que el sistema trabaje con esos nuevos datos. \\
\hline
\end{tabular}

\vspace{1cm}


\begin{tabular}{|m{4pt}|m{7.33cm}|m{4pt}|m{7.33cm}|}
\hline
\multicolumn{4}{|c|}{\textbf{Curso normal}} \\
\hline
\textbf{1} & El oficinista inicia el proceso de modificación. & \textbf{2} & El sistema solicita los datos de la furgoneta. \\
\hline
\textbf{3} & El oficinista introduce los datos. & \textbf{4} & El sistema comprueba los datos. \\
\hline
\textbf{5} & El oficinista introduce las modificaciones. & \textbf{6} & El sistema solicita confirmación \\
\hline
\textbf{7} & El oficinista confirma los datos. & \textbf{8} & El sistema almacena los nuevos datos. \\
\hline
\end{tabular}

\vspace{1cm}

\begin{tabular}{|m{10pt}|m{7.15cm}|m{10pt}|m{7.15cm}|}
\hline
\multicolumn{4}{|m{16.2cm}|}{\textbf{Cursos alternos}} \\
\hline
\multicolumn{4}{|m{16.2cm}|}{\textbf{4a) Los datos no son válidos}} \\
\hline
\multicolumn{4}{|m{16.2cm}|}{El sistema informa del error y solicita que los datos se vuelvan a introducir.} \\
\hline
\end{tabular}

\vspace{1cm}

\begin{tabular}{|m{3.72cm}|m{3.72cm}|m{3.72cm}|m{3.72cm}|}
\hline
\multicolumn{4}{|c|}{\textbf{Otros datos}} \\
\hline
\textbf{Frecuencia esperada} & Media (una vez al trimestre) & \textbf{Rendimiento} & - \\
\hline
\textbf{Importancia} & Media & \textbf{Urgencia} & Media \\
\hline
\textbf{Estado} & En desarrollo & \textbf{Estabilidad} & Alta \\
\hline
\end{tabular}

\vspace{1cm}

\begin{tabular}{|m{16.2cm}|}
\hline
\textbf{Comentarios} \\
\hline
- \\
\hline
\end{tabular}

\caption{Modificación furgoneta}

\end{table}


%%%%%%%%%ESTABLECER DISPONIBILIDAD FURGONETA%%%%%%%%%

\begin{table}[H]
\centering
\begin{tabular}{|m{3cm}|m{4cm}|m{2cm}|m{2cm}|m{2cm}|m{1cm}|}
\hline
\textbf{Caso de uso} &  \multicolumn{4}{m{8cm}|}{Establecer disponibilidad furgoneta} \vline &  \cellcolor{gray!40}CU\arabic{contadorCU}  \stepcounter{contadorCU}
\\
\hline
\textbf{Actores} & \multicolumn{5}{m{8cm}|}{Oficinista} \\
\hline
\textbf{Tipo} & \multicolumn{5}{m{8cm}|}{Real} \\
\hline
\textbf{Referencias} &\multicolumn{5}{m{8cm}|}{-} \\
\hline
\textbf{Precondición} & \multicolumn{5}{m{8cm}|}{La furgoneta debe estar dada de alta en el sistema.} \\
\hline
\textbf{Postcondición} & \multicolumn{5}{m{8cm}|}{-} \\
\hline
\textbf{Autor} & Carlos Sánchez Páez & \textbf{Fecha} & 08/04/2018 & \textbf{Versión} & 2.0 \\
\hline
\end{tabular}

\vspace{1cm}

\begin{tabular}{|m{16.2cm}|}
\hline
\textbf{Propósito} \\
\hline
Establecer el horario de una furgoneta. \\
\hline
\end{tabular}

\vspace{1cm}

\begin{tabular}{|m{16.2cm}|}
\hline
\textbf{Resumen} \\
\hline
Informar al sistema de la disponibilidad de una furgoneta en concreto en un momento determinado. \\
\hline
\end{tabular}

\vspace{1cm}


\begin{tabular}{|m{4pt}|m{7.33cm}|m{4pt}|m{7.33cm}|}
\hline
\multicolumn{4}{|c|}{\textbf{Curso normal}} \\
\hline
\textbf{1} & El oficinista pide iniciar el proceso. & \textbf{2} & El sistema solicita los datos de la furgoneta. \\
\hline
\textbf{3} & El oficinista introduce los datos. & \textbf{4} & El sistema comprueba los datos. \\
\hline
\textbf{5} & El oficinista introduce los datos de disponibilidad. & \textbf{6} & El sistema solicita confirmación \\
\hline
\textbf{7} & El oficinista confirma los datos. & \textbf{8} & El sistema almacena los datos. \\
\hline
\end{tabular}

\vspace{1cm}

\begin{tabular}{|m{10pt}|m{7.15cm}|m{10pt}|m{7.15cm}|}
\hline
\multicolumn{4}{|m{16.2cm}|}{\textbf{Cursos alternos}} \\
\hline
\multicolumn{4}{|m{16.2cm}|}{\textbf{4a) Los datos no son válidos}} \\
\hline
\multicolumn{4}{|m{16.2cm}|}{El sistema informa del error y solicita que los datos se vuelvan a introducir.} \\
\hline
\end{tabular}

\vspace{1cm}

\begin{tabular}{|m{3.72cm}|m{3.72cm}|m{3.72cm}|m{3.72cm}|}
\hline
\multicolumn{4}{|c|}{\textbf{Otros datos}} \\
\hline
\textbf{Frecuencia esperada} & Alta (a diario) & \textbf{Rendimiento} & - \\
\hline
\textbf{Importancia} & Extrema & \textbf{Urgencia} & Máxima \\
\hline
\textbf{Estado} & En desarrollo & \textbf{Estabilidad} & Alta \\
\hline
\end{tabular}

\vspace{1cm}

\begin{tabular}{|m{16.2cm}|}
\hline
\textbf{Comentarios} \\
\hline
- \\
\hline
\end{tabular}

\caption{Establecer disponibilidad furgoneta}

\end{table}

%%%%%%%%%%%CONSULTAR ESTADO FURGONETA%%%%%%%%%

\begin{table}[H]
\centering
\begin{tabular}{|m{3cm}|m{4cm}|m{2cm}|m{2cm}|m{2cm}|m{1cm}|}
\hline
\textbf{Caso de uso} &  \multicolumn{4}{m{8cm}|}{Consultar estado furgoneta} \vline &  \cellcolor{gray!40}CU\arabic{contadorCU}  \stepcounter{contadorCU}
\\
\hline
\textbf{Actores} & \multicolumn{5}{m{8cm}|}{Oficinista} \\
\hline
\textbf{Tipo} & \multicolumn{5}{m{8cm}|}{Real} \\
\hline
\textbf{Referencias} &\multicolumn{5}{m{8cm}|}{-} \\
\hline
\textbf{Precondición} & \multicolumn{5}{m{8cm}|}{La furgoneta debe estar dada de alta en el sistema.} \\
\hline
\textbf{Postcondición} & \multicolumn{5}{m{8cm}|}{-} \\
\hline
\textbf{Autor} & Carlos Sánchez Páez & \textbf{Fecha} & 08/04/2018 & \textbf{Versión} & 2.0 \\
\hline
\end{tabular}

\vspace{1cm}

\begin{tabular}{|m{16.2cm}|}
\hline
\textbf{Propósito} \\
\hline
Consultar el estado de una furgoneta. \\
\hline
\end{tabular}

\vspace{1cm}

\begin{tabular}{|m{16.2cm}|}
\hline
\textbf{Resumen} \\
\hline
Ver información relativa a una furgoneta en un momento determinado. \\
\hline
\end{tabular}

\vspace{1cm}

\begin{tabular}{|m{4pt}|m{7.33cm}|m{4pt}|m{7.33cm}|}
\hline
\multicolumn{4}{|c|}{\textbf{Curso normal}} \\
\hline
\textbf{1} & El oficinista pide iniciar el proceso. & \textbf{2} & El sistema solicita los datos de la furgoneta. \\
\hline
\textbf{3} & El oficinista introduce los datos. & \textbf{4} & El sistema comprueba los datos. \\
\hline
\textbf{5} &  & \textbf{6} & El sistema muestra los datos de disponibilidad. \\
\hline
\end{tabular}

\vspace{1cm}

\begin{tabular}{|m{10pt}|m{7.15cm}|m{10pt}|m{7.15cm}|}
\hline
\multicolumn{4}{|m{16.2cm}|}{\textbf{Cursos alternos}} \\
\hline
\multicolumn{4}{|m{16.2cm}|}{\textbf{4a) Los datos no son válidos}} \\
\hline
\multicolumn{4}{|m{16.2cm}|}{El sistema informa del error y solicita que los datos se vuelvan a introducir.} \\
\hline
\end{tabular}

\vspace{1cm}

\begin{tabular}{|m{3.72cm}|m{3.72cm}|m{3.72cm}|m{3.72cm}|}
\hline
\multicolumn{4}{|c|}{\textbf{Otros datos}} \\
\hline
\textbf{Frecuencia esperada} & Alta (a diario) & \textbf{Rendimiento} & - \\
\hline
\textbf{Importancia} & Extrema & \textbf{Urgencia} & Máxima \\
\hline
\textbf{Estado} & En desarrollo & \textbf{Estabilidad} & Alta \\
\hline
\end{tabular}

\vspace{1cm}

\begin{tabular}{|m{16.2cm}|}
\hline
\textbf{Comentarios} \\
\hline
- \\
\hline
\end{tabular}

\caption{Consultar estado furgoneta}

\end{table}



%%%%%%%%CONSULTAR ESTADO PAQUETES%%%%%%%

\begin{table}[H]
\centering
\begin{tabular}{|m{3cm}|m{4cm}|m{2cm}|m{2cm}|m{2cm}|m{1cm}|}
\hline
\textbf{Caso de uso} &  \multicolumn{4}{m{8cm}|}{Consultar estado paquetes} \vline &  \cellcolor{gray!40}CU\arabic{contadorCU}  \stepcounter{contadorCU}
\\
\hline
\textbf{Actores} & \multicolumn{5}{m{8cm}|}{Oficinista} \\
\hline
\textbf{Tipo} & \multicolumn{5}{m{8cm}|}{Real} \\
\hline
\textbf{Referencias} &\multicolumn{5}{m{8cm}|}{-} \\
\hline
\textbf{Precondición} & \multicolumn{5}{m{8cm}|}{El paquete debe estar dado de alta en el sistema.} \\
\hline
\textbf{Postcondición} & \multicolumn{5}{m{8cm}|}{-} \\
\hline
\textbf{Autor} & Carlos Sánchez Páez & \textbf{Fecha} & 08/04/2018 & \textbf{Versión} & 2.0 \\
\hline
\end{tabular}

\vspace{1cm}

\begin{tabular}{|m{16.2cm}|}
\hline
\textbf{Propósito} \\
\hline
Ver el estado de un paquete. \\
\hline
\end{tabular}

\vspace{1cm}

\begin{tabular}{|m{16.2cm}|}
\hline
\textbf{Resumen} \\
\hline
Obtener datos relativos a un paquete (ubicación, fecha de entrega, conductor asociado, etc.) \\
\hline
\end{tabular}

\vspace{1cm}

\begin{tabular}{|m{4pt}|m{7.33cm}|m{4pt}|m{7.33cm}|}
\hline
\multicolumn{4}{|c|}{\textbf{Curso normal}} \\
\hline
\textbf{1} & El oficinista pide iniciar el proceso. & \textbf{2} & El sistema solicita un identificador de paquete. \\
\hline
\textbf{3} & El oficinista introduce el identificador. & \textbf{4} & El sistema comprueba el identificador. \\
\hline
\textbf{5} &  & \textbf{6} & El sistema muestra la información. \\
\hline
\end{tabular}

\vspace{1cm}

\begin{tabular}{|m{10pt}|m{7.15cm}|m{10pt}|m{7.15cm}|}
\hline
\multicolumn{4}{|m{16.2cm}|}{\textbf{Cursos alternos}} \\
\hline
\multicolumn{4}{|m{16.2cm}|}{\textbf{4a) Los datos no son válidos}} \\
\hline
\multicolumn{4}{|m{16.2cm}|}{El sistema informa del error y solicita que los datos se vuelvan a introducir.} \\
\hline
\end{tabular}
\vspace{1cm}

\begin{tabular}{|m{3.72cm}|m{3.72cm}|m{3.72cm}|m{3.72cm}|}
\hline
\multicolumn{4}{|c|}{\textbf{Otros datos}} \\
\hline
\textbf{Frecuencia esperada} & Alta (a diario) & \textbf{Rendimiento} & - \\
\hline
\textbf{Importancia} & Extrema & \textbf{Urgencia} & Máxima \\
\hline
\textbf{Estado} & En desarrollo & \textbf{Estabilidad} & Alta \\
\hline
\end{tabular}

\vspace{1cm}

\begin{tabular}{|m{16.2cm}|}
\hline
\textbf{Comentarios} \\
\hline
- \\
\hline
\end{tabular}

\caption{Consultar estado paquetes}

\end{table}

%%%%DEPOSITAR PAQUETE A ENVIAR%%%%%%%%

\begin{table}[H]
\centering
\begin{tabular}{|m{3cm}|m{4cm}|m{2cm}|m{2cm}|m{2cm}|m{1cm}|}
\hline
\textbf{Caso de uso} &  \multicolumn{4}{m{8cm}|}{Depositar paquete a enviar} \vline &  \cellcolor{gray!40}CU\arabic{contadorCU}  \stepcounter{contadorCU}
\\
\hline
\textbf{Actores} & \multicolumn{5}{m{8cm}|}{Oficinista y almacenero.} \\
\hline
\textbf{Tipo} & \multicolumn{5}{m{8cm}|}{Real} \\
\hline
\textbf{Referencias} &\multicolumn{5}{m{8cm}|}{-} \\
\hline
\textbf{Precondición} & \multicolumn{5}{m{8cm}|}{-} \\
\hline
\textbf{Postcondición} & \multicolumn{5}{m{8cm}|}{-} \\
\hline
\textbf{Autor} & Carlos Sánchez Páez & \textbf{Fecha} & 08/04/2018 & \textbf{Versión} & 2.0 \\
\hline
\end{tabular}

\vspace{1cm}

\begin{tabular}{|m{16.2cm}|}
\hline
\textbf{Propósito} \\
\hline
Llevar un paquete de una oficina a un almacén. \\
\hline
\end{tabular}

\vspace{1cm}

\begin{tabular}{|m{16.2cm}|}
\hline
\textbf{Resumen} \\
\hline
El paquete es transportado desde una oficina a un almacén para proceder a su transporte. \\
\hline
\end{tabular}

\vspace{1cm}

\begin{tabular}{|m{4pt}|m{7.33cm}|m{4pt}|m{7.33cm}|}
\hline
\multicolumn{4}{|c|}{\textbf{Curso normal}} \\
\hline
\textbf{1} & El oficinista pide iniciar el proceso. & \textbf{2} & El sistema solicita los datos del paquete. \\
\hline
\textbf{3} & El oficinista introduce los datos. & \textbf{4} & El sistema comprueba los datos. \\
\hline
\textbf{5} & El oficinista introduce el almacén de destino. & \textbf{6} & El sistema solicita confirmación \\
\hline
\textbf{7} & El oficinista confirma los datos. & \textbf{8} & El sistema almacena los datos.\\
\hline
\end{tabular}

\vspace{1cm}

\begin{tabular}{|m{10pt}|m{7.15cm}|m{10pt}|m{7.15cm}|}
\hline
\multicolumn{4}{|m{16.2cm}|}{\textbf{Cursos alternos}} \\
\hline
\multicolumn{4}{|m{16.2cm}|}{\textbf{4a) Los datos no son válidos}} \\
\hline
\multicolumn{4}{|m{16.2cm}|}{El sistema informa del error y solicita que los datos se vuelvan a introducir.} \\
\hline
\end{tabular}

\vspace{1cm}

\begin{tabular}{|m{3.72cm}|m{3.72cm}|m{3.72cm}|m{3.72cm}|}
\hline
\multicolumn{4}{|c|}{\textbf{Otros datos}} \\
\hline
\textbf{Frecuencia esperada} & Alta (a diario) & \textbf{Rendimiento} & - \\
\hline
\textbf{Importancia} & Extrema & \textbf{Urgencia} & Máxima \\
\hline
\textbf{Estado} & En desarrollo & \textbf{Estabilidad} & Alta \\
\hline
\end{tabular}

\vspace{1cm}

\begin{tabular}{|m{16.2cm}|}
\hline
\textbf{Comentarios} \\
\hline
- \\
\hline
\end{tabular}

\caption{Depositar paquete a enviar}

\end{table}

%%%%%%%%%%%%%%%%%%%%%%%%%%%%Fin del documento%%%%%%%%%%%%%%%%%%%%%%%%%%%%%%%%%
\end{document}
