\documentclass[12pt,spanish]{article}
\usepackage[spanish]{babel}
\usepackage{graphicx}
\usepackage{enumitem}
\usepackage[hidelinks]{hyperref}
\usepackage[table]{xcolor}
\usepackage{multirow}
\usepackage{float}
\usepackage{array}
\graphicspath{ {../../LaTeX/img/} {/home/csp98/latex/img/}}
\selectlanguage{spanish}
\usepackage[utf8]{inputenc}
\usepackage{graphicx}
\usepackage[a4paper,left=3cm,right=2cm,top=2.5cm,bottom=2.5cm]{geometry}
\makeindex

\begin{document}
\begin{titlepage}

\newlength{\centeroffset}
\setlength{\centeroffset}{-0.5\oddsidemargin}
\addtolength{\centeroffset}{0.5\evensidemargin}
\thispagestyle{empty}

\noindent\hspace*{\centeroffset}\begin{minipage}{\textwidth}

\centering
\includegraphics[width=0.9\textwidth]{logo_ugr.jpg}\\[1.4cm]

\textsc{ \Large Fundamentos de Ingeniería del Software\\[0.2cm]}
\textsc{GRADO EN INGENIERÍA INFORMÁTICA}\\[1cm]

{\Huge\bfseries Relación 2.2\\
}
\noindent\rule[-1ex]{\textwidth}{3pt}\\[3.5ex]
{\large\bfseries Ejercicio 2}
\end{minipage}

\vspace{2.5cm}
\noindent\hspace*{\centeroffset}
\begin{minipage}{\textwidth}
\centering

\textbf{Autor}\\ {Carlos Sánchez Páez}\\[2.5ex]
\includegraphics[width=0.3\textwidth]{etsiit_logo.png}\\[0.1cm]
\vspace{1.5cm}
\includegraphics[width=0.2\textwidth]{lsi.png}\\[0.1cm]
\vspace{1cm}
\textsc{Escuela Técnica Superior de Ingenierías Informática y de Telecomunicación}\\
\vspace{1cm}
\textsc{Curso 2017-2018}
\end{minipage}
\end{titlepage}
%%%%%%%%%%%%%%%%%%%%%%%%Comienzo del documento%%%%%%%%%%%%%%%%%%%%%%%%%%%%%%%
\tableofcontents
\thispagestyle{empty}
\listoftables
\newpage
\setcounter{page}{1}
\section{Enunciado}
{\large\textbf{Ejercicio 7.} Utilizando su conocimiento del funcionamiento de los cajeros automáticos,proponga un conjunto de casos de uso, para el actor cliente y elabore sus descripciones básicas.}
\section{Resolución}
\subsection{Lista de casos de uso}

Los casos de uso del cajero automático serían los siguientes:
\begin{enumerate}[label=\textbf{CU-\arabic*}]
\item Sacar dinero.
\item Ingresar dinero.
\item Consultar movimientos.
\item Cambiar PIN de la tarjeta/libreta.
\item Realizar transferencia.
\end{enumerate}
Procedemos a su descripción general:

\subsection{Sacar dinero}

\begin{table}[H]
\centering
\begin{tabular}{|m{3cm}|m{4cm}|m{2cm}|m{2cm}|m{2cm}|m{1cm}|}
\hline
\textbf{Caso de uso} &  \multicolumn{4}{m{8cm}|}{Sacar dinero} \vline &  \cellcolor{gray!40}CU1 \\
\hline
\textbf{Actores} & \multicolumn{5}{m{8cm}|}{Cliente} \\
\hline
\textbf{Tipo} & \multicolumn{5}{m{8cm}|}{Primario} \\
\hline
\textbf{Referencias} &\multicolumn{5}{m{8cm}|}{-} \\
\hline
\textbf{Precondición} & \multicolumn{5}{m{8cm}|}{Que el cliente tenga saldo en cuenta.} \\
\hline
\textbf{Postcondición} & \multicolumn{5}{m{8cm}|}{El saldo de la cuenta del cliente se ve reducido.} \\
\hline
\textbf{Autor} & Carlos Sánchez Páez & \textbf{Fecha} & 20/03/2018 & \textbf{Versión} & 2.0 \\
\hline
\end{tabular}

\vspace{1cm}

\begin{tabular}{|m{16.2cm}|}
\hline
\textbf{Propósito} \\
\hline
Que el cliente pueda sacar parte (o todo) el dinero en efectivo que tiene en su cuenta de ahorros. \\
\hline
\end{tabular}

\vspace{1cm}

\begin{tabular}{|m{16.2cm}|}
\hline
\textbf{Resumen} \\
\hline
El cliente solicita que se le de en efectivo una determinada cantidad de dinero que debe ser menor o igual al saldo en cuenta y/o créditos asociados. \\
\hline
\end{tabular}

\caption{Sacar dinero}

\end{table}

\subsection{Ingresar dinero} 


\begin{table}[H]
\centering
\begin{tabular}{|m{3cm}|m{4cm}|m{2cm}|m{2cm}|m{2cm}|m{1cm}|}
\hline
\textbf{Caso de uso} &  \multicolumn{4}{m{8cm}|}{Ingresar dinero} \vline &  \cellcolor{gray!40}CU2 \\
\hline
\textbf{Actores} & \multicolumn{5}{m{8cm}|}{Cliente} \\
\hline
\textbf{Tipo} & \multicolumn{5}{m{8cm}|}{Primario} \\
\hline
\textbf{Referencias} &\multicolumn{5}{m{8cm}|}{-} \\
\hline
\textbf{Precondición} & \multicolumn{5}{m{8cm}|}{Que el cliente tenga dinero en efectivo.} \\
\hline
\textbf{Postcondición} & \multicolumn{5}{m{8cm}|}{El saldo de la cuenta del cliente se ve incrementado.} \\
\hline
\textbf{Autor} & Carlos Sánchez Páez & \textbf{Fecha} & 20/03/2018 & \textbf{Versión} & 2.0 \\
\hline
\end{tabular}

\vspace{1cm}

\begin{tabular}{|m{16.2cm}|}
\hline
\textbf{Propósito} \\
\hline
Que el cliente pueda ingresar parte (o todo) el dinero en efectivo que tiene físicamente en ese momento. \\
\hline
\end{tabular}

\vspace{1cm}

\begin{tabular}{|m{16.2cm}|}
\hline
\textbf{Resumen} \\
\hline
El cliente solicita que se ingrese el dinero en efectivo que posee a una de sus cuentas de ahorros. \\
\hline
\end{tabular}

\caption{Ingresar dinero}

\end{table}

\subsection{Consultar movimientos}


\begin{table}[H]
\centering
\begin{tabular}{|m{3cm}|m{4cm}|m{2cm}|m{2cm}|m{2cm}|m{1cm}|}
\hline
\textbf{Caso de uso} &  \multicolumn{4}{m{8cm}|}{Consultar movimientos} \vline &  \cellcolor{gray!40}CU3 \\
\hline
\textbf{Actores} & \multicolumn{5}{m{8cm}|}{Cliente} \\
\hline
\textbf{Tipo} & \multicolumn{5}{m{8cm}|}{Primario} \\
\hline
\textbf{Referencias} &\multicolumn{5}{m{8cm}|}{-} \\
\hline
\textbf{Precondición} & \multicolumn{5}{m{8cm}|}{-} \\
\hline
\textbf{Postcondición} & \multicolumn{5}{m{8cm}|}{-} \\
\hline
\textbf{Autor} & Carlos Sánchez Páez & \textbf{Fecha} & 20/03/2018 & \textbf{Versión} & 2.0 \\
\hline
\end{tabular}

\vspace{1cm}

\begin{tabular}{|m{16.2cm}|}
\hline
\textbf{Propósito} \\
\hline
Que el cliente pueda visualizar los movimientos de su cuenta de ahorros en un período determinado. \\
\hline
\end{tabular}

\vspace{1cm}

\begin{tabular}{|m{16.2cm}|}
\hline
\textbf{Resumen} \\
\hline
El cliente obtiene (en papel o en pantalla) un listado de los últimos movimientos de una de sus cuentas de ahorros (ingresos, retiradas, pagos, etc.) \\
\hline
\end{tabular}

\caption{Consultar movimientos}

\end{table}


\subsection{Cambiar PIN de la tarjeta/libreta}


\begin{table}[H]
\centering
\begin{tabular}{|m{3cm}|m{4cm}|m{2cm}|m{2cm}|m{2cm}|m{1cm}|}
\hline
\textbf{Caso de uso} &  \multicolumn{4}{m{8cm}|}{Cambiar PIN} \vline &  \cellcolor{gray!40}CU4 \\
\hline
\textbf{Actores} & \multicolumn{5}{m{8cm}|}{Cliente} \\
\hline
\textbf{Tipo} & \multicolumn{5}{m{8cm}|}{Primario} \\
\hline
\textbf{Referencias} &\multicolumn{5}{m{8cm}|}{-} \\
\hline
\textbf{Precondición} & \multicolumn{5}{m{8cm}|}{El cliente debe recordar el PIN anterior.} \\
\hline
\textbf{Postcondición} & \multicolumn{5}{m{8cm}|}{El PIN de la tarjeta/libreta cambia.} \\
\hline
\textbf{Autor} & Carlos Sánchez Páez & \textbf{Fecha} & 20/03/2018 & \textbf{Versión} & 2.0 \\
\hline
\end{tabular}

\vspace{1cm}

\begin{tabular}{|m{16.2cm}|}
\hline
\textbf{Propósito} \\
\hline
Que el cliente pueda cambiar el PIN (\emph{Personal Identification Number}) de una de sus libretas o tarjetas. \\
\hline
\end{tabular}

\vspace{1cm}

\begin{tabular}{|m{16.2cm}|}
\hline
\textbf{Resumen} \\
\hline
El cliente solicita que el PIN para poder realizar operaciones con su tarjeta/libreta cambie por uno nuevo que debe ser diferente al anterior. \\
\hline
\end{tabular}

\caption{Cambiar PIN}

\end{table}


\subsection{Realizar transferencia}


\begin{table}[H]
\centering
\begin{tabular}{|m{3cm}|m{4cm}|m{2cm}|m{2cm}|m{2cm}|m{1cm}|}
\hline
\textbf{Caso de uso} &  \multicolumn{4}{m{11cm}|}{Realizar transferencia} \vline &  \cellcolor{gray!40}CU5 \\
\hline
\textbf{Actores} & \multicolumn{5}{m{8cm}|}{Cliente} \\
\hline
\textbf{Tipo} & \multicolumn{5}{m{11cm}|}{Primario} \\
\hline
\textbf{Referencias} &\multicolumn{5}{m{11cm}|}{-} \\
\hline
\textbf{Precondición} & \multicolumn{5}{m{11cm}|}{Que el cliente tenga saldo suficiente en cuenta.} \\
\hline
\textbf{Postcondición} & \multicolumn{5}{m{11cm}|}{El saldo de la cuenta del cliente se ve reducido.} \\
\hline
\textbf{Autor} & Carlos Sánchez Páez & \textbf{Fecha} & 20/03/2018 & \textbf{Versión} & 2.0 \\
\hline
\end{tabular}

\vspace{1cm}

\begin{tabular}{|m{16.2cm}|}
\hline
\textbf{Propósito} \\
\hline
Que el cliente pueda transferir parte (o todo) el dinero que tiene en su cuenta de ahorros a otra persona. \\
\hline
\end{tabular}

\vspace{1cm}

\begin{tabular}{|m{16.2cm}|}
\hline
\textbf{Resumen} \\
\hline
El cliente solicita que se transfiera una cantidad de dinero de su cuenta a otra entidad, que puede ser cliente del banco o no (de otro banco distinto). \\
\hline
\end{tabular}

\caption{Realizar transferencia}

\end{table}


 

%%%%%%%%%%%%%%%%%%%%%%%%%%%%Fin del documento%%%%%%%%%%%%%%%%%%%%%%%%%%%%%%%%%
\end{document}
