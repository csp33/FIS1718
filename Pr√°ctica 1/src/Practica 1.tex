\documentclass[12pt,spanish]{article}
\usepackage[spanish]{babel}
\selectlanguage{spanish}
\usepackage[utf8]{inputenc}
\usepackage{float}
\usepackage{array}
\usepackage{enumerate}
\usepackage{multirow}
\usepackage{titling}
\usepackage{graphicx}
\usepackage{enumitem}
\usepackage{blindtext}
\usepackage{enumitem}
\usepackage[a4paper,left=2.5cm,right=2.5cm,top=2.5cm,bottom=2.5cm]{geometry}
%%%%%%%%%%%%%%%%%%%%%%%%%%%%%%%%%%%%%%%%%%%%%%%%%%%%%%%%%%%%%%%%%%%%%%%%%%%%%
\title{Sistema de gestión de transportes \\  \small{para Envíamelo S.A.}}
\date{\today}
\author{Carlos Sánchez Páez \\ José Miguel Pelegrina Pelegrina \\ José Baena Cobos}
\setlength{\droptitle}{10em}
%%%%%%%%%%%%%%%%%%%%%%%%%%%%%%%%%%%%%%%%%%%%%%%%%%%%%%%%%%%%%%%%%%%%%%%%%%%%
\begin{document}
	
\maketitle
\thispagestyle{empty}
\newpage
\tableofcontents
\listoftables
\thispagestyle{empty}
\newpage
\setcounter{page}{1}

\section{Dominio del problema y descripción general del sistema}
Este proyecto tiene como objetivo desarrollar un software para sustituir al actual sistema de gestión de la empresa logística Envíamelo S.A. Ésta dispone de una flota de furgonetas para la distribución de paquetes por todo el territorio español. En cada capital de provincia hay una oficina y un almacén para la carga y descarga de los paquetes que serán transportados de un almacén de origen a otro de destino. Además, cada almacén dispone de un conjunto de pequeñas furgonetas que se encargan de la recogida de paquetes de los usuarios que quieren enviar un paquete, así como de su reparto al destinatario final. \\
Este nuevo sistema busca agilizar los trámites tanto en la recogida, como el envío y recepción del paquete; a la vez que facilitar a la empresa la gestión de sus recursos. Con ésto se pretende aumentar el margen de beneficios de la empresa.

\subsection{Objetivos}

\subsubsection{Objetivos orientados al cliente}

\begin{enumerate}[label=\textbf{OBJ-\arabic*}]
	\item Permitir que el usuario pueda obtener información relativa al servicio contratado en todo momento.
	\item Permitir que el usuario concrete la franja temporal más adecuada para el envío y la recepción del producto.
	\item Permitir que el usuario pueda gestionar el procedimiento a seguir en caso de ausencia del destinatario.
	\item Permitir que el usuario pueda obtener un presupuesto de un determinado servicio.
	\item Permitir al usuario la localización de las oficinas más cercanas a su ubicación
	\item Permitir a un usuario registrado la valoración del servicio ofrecido por la empresa, con el objetivo de mejorar el mismo.
\end{enumerate}

\subsubsection{Objetivos orientados a la empresa}
\begin{enumerate}[resume , label=\textbf{OBJ-\arabic*}]
	\item Gestionar los almacenes.
	\item Optimizar las ruta de transporte y reparto para obtener el mayor rendimiento posible. 
	\item Obtener información sobre la flota de furgonetas de transporte y de reparto.
	\item Conocer la disponibilidad de los conductores en tiempo real.
	\item Gestionar la opinión de los clientes para poder mejorar el servicio.
\end{enumerate}
\newpage
\section{Lista de requisitos}

\subsection{Descripción de implicados y usuarios finales}

\subsubsection{Entorno de usuario}
Los usuarios finales del sistema a desarrollar serán dos: el cliente, que contratará un determinado producto y el administrador de sistema, que se encargará de configurar el catálogo de servicios disponibles. El primero de ellos no tiene por qué tener un amplio conocimiento informático, pero el administrador si habrá de tener nociones de usuario medio.
\subsubsection{Resumen de implicados}

\begin{table}[H]
\begin{center}
\begin{tabular}{|p{3cm}|p{4cm}|p{3cm}|p{4cm}|}
\hline
\textbf{Nombre} & \textbf{Descripción} & \textbf{Tipo} & \textbf{Responsabilidad} \\
\hline
Emisor & Representa un potencial socio & Usuario sistema & Enviar un paquete \\
\hline
Socio & Representa un socio & Usuario sistema & Enviar un paquete. \\
\hline
Receptor & Representa un potencial cliente & Usuario sistema & Recibir un paquete \\
\hline
Empleado de almacén & Representa un empleado & Usuario producto & Gestiona los paquetes dentro del almacén \\
\hline
Repartidor & Representa un empleado & Usuario producto & Hace llegar el paquete desde la oficina hasta su destino \\
\hline
Encargado de oficina & Representa un empleado & Usuario producto & Dirige a los empleados de cada almacén.  \\
\hline
\end{tabular}
\caption{Resumen de implicados}
\end{center}
\end{table}

\subsubsection{Perfiles de los implicados}
%%%%%%%%%%%%%%%%%%%%%EMISOR%%%%%%%%%%%%%%%%%%%%%
\begin{table}[H]
\begin{center}
\begin{tabular}{|l|m{11cm}|}
\hline
\textbf{Descripción} & Emisor (potencial socio) \\
\hline
\textbf{Tipo} & Utiliza el sistema para contratar un envío y acordar la recogida del paquete. \\
\hline
\textbf{Responsabilidades} & 
\begin{minipage}{11cm}
    \vskip 1pt
    \begin{enumerate}
   		\item Contratar servicios.
     	\item Monitorizar el estado de su envío.
  		\item Consultar el estado de los servicios contratados.
   \end{enumerate}
   \vskip 1pt
 \end{minipage}\\ 
\hline
\textbf{Criterios de éxito} & Que el sistema le permita realizar sus gestiones de forma fácil y eficiente.\\
\hline
\textbf{Implicación} & Utilizará el sistema para enviar paquetes de forma esporádica o bien para hacerse socio. \\
\hline
\end{tabular}
\caption{Emisor}
\end{center}
\end{table}

%%%%%%%%%%%%%%%%%%%%%DESTINATARIO%%%%%%%%%%%%%%%%%%%%%

\begin{table}[H]
\begin{center}
\begin{tabular}{|l|m{11cm}|}
\hline
\textbf{Descripción} & Destinatario (potencial cliente) \\
\hline
\textbf{Tipo} & No utiliza el sistema de forma directa, sino que juega un papel fundamental en el desarrollo de un envío. \\
\hline
\textbf{Responsabilidades} & 
\begin{minipage}{11cm}
    \vskip 1pt
    \begin{enumerate}
   \item Recibir envíos.
   \item Monitorizar el estado de su envío.
   \end{enumerate}
   \vskip 1pt
 \end{minipage}\\ 
\hline
\textbf{Criterios de éxito} & Que el sistema le permita realizar sus consultas y gestiones de forma fácil y eficiente.\\
\hline
\textbf{Implicación} & Utilizará el sistema para obtener información relativa al servicio del que es receptor. \\
\hline
\end{tabular}
\caption{Destinatario}
\end{center}
\end{table}

%%%%%%%%%%%%%%%%%%%%%SOCIO%%%%%%%%%%%%%%%%%%%%%

\begin{table}[H]
\begin{center}
\begin{tabular}{|l|m{11cm}|}
\hline
\textbf{Descripción} & Socio \\
\hline
\textbf{Tipo} & Utiliza el sistema de forma frecuente para contratar envíos y acordar su recogida. \\
\hline
\textbf{Responsabilidades} & 
\begin{minipage}{11cm}
    \vskip 1pt
    \begin{enumerate}
   		\item Contratar servicios.
   		\item Monitorizar el estado de su envío.
     	\item Valorar el trato y la calidad del servicio ofrecido.
   \end{enumerate}
   \vskip 1pt
 \end{minipage}\\ 
\hline
\textbf{Criterios de éxito} & Que el sistema le permita realizar sus consultas y gestiones de forma fácil y eficiente.\\
\hline
\textbf{Implicación} & Utilizará el sistema para contratar servicios y valorar a la empresa. \\
\hline
\end{tabular}
\caption{Socio}
\end{center}
\end{table}

%%%%%%%%%%%%%%%%%%%%%EMPLEADO DE ALMACÉN%%%%%%%%%%%%%%%%%%%%%

\begin{table}[H]
\begin{center}
\begin{tabular}{|l|m{9.5cm}|}
\hline
\textbf{Descripción} & Empleado de almacén \\
\hline
\textbf{Grado de responsabilidad} & Media. \\
\hline
\textbf{Responsabilidades} & 
\begin{minipage}{9.5cm}
    \vskip 1pt
    \begin{enumerate}
   		\item Cargar y descargar los paquetes de los correspondientes vehículos.
   		\item Comprobar que no haya daños materiales en los productos.
   \end{enumerate}
   \vskip 1pt
 \end{minipage}\\ 
\hline
\textbf{Criterios de éxito} & Debe realizar una correcta distribución de los paquetes.\\
\hline
\textbf{Implicación} & Se encarga de que los paquetes estén en el vehículo idóneo. \\
\hline
\end{tabular}
\caption{Empleado de almacén}
\end{center}
\end{table}

%%%%%%%%%%%%%%%%%%%%%REPARTIDOR%%%%%%%%%%%%%%%%%%%%%
\begin{table}[H]
\begin{center}
\begin{tabular}{|l|m{9.5cm}|}
\hline
\textbf{Descripción} & Repartidor \\
\hline
\textbf{Grado de responsabilidad} & Media \\
\hline
\textbf{Responsabilidades} & 
\begin{minipage}{9.5cm}
    \vskip 1pt
    \begin{enumerate}
   		\item Entregar los paquetes a los destinatarios.
   		\item Recoger los paquetes de los emisores.
   		\item Comprobar que no haya daños materiales en los productos.
   		\item Anotar requisitos específicos de los emisores (franja horaria preferida, etc.)
   \end{enumerate}
   \vskip 1pt
 \end{minipage}\\ 
\hline
\textbf{Criterios de éxito} & Debe realizar un correcto transporte de los paquetes.\\
\hline
\textbf{Implicación} & Se encarga de que los paquetes recorran la ruta correcta. \\
\hline
\end{tabular}
\caption{Repartidor}
\end{center}
\end{table}

%%%%%%%%%%%%%%%%%%%%%ENCARGADO DE OFICINA%%%%%%%%%%%%%%%%%%%%%

\begin{table}[H]
\begin{center}
\begin{tabular}{|l|m{9.5cm}|}
\hline
\textbf{Descripción} & Encargado de oficina \\
\hline
\textbf{Grado de responsabilidad} & Media. \\
\hline
\textbf{Responsabilidades} & 
\begin{minipage}{9.5cm}
    \vskip 1pt
    \begin{enumerate}
   		\item Ayudar a los clientes en la elección del mejor servicio acorde a sus necesidades.
   		\item Ofrecer ayuda a los clientes.
   		\item Realizar la contratación del servicio.
   		\item Tramitar la afiliación de un cliente a la empresa.
   		\item Resolver los problemas que el cliente pueda tener.
   \end{enumerate}
   \vskip 1pt
 \end{minipage}\\ 
\hline
\textbf{Criterios de éxito} & Debe realizar una correcta atención al cliente.\\
\hline
\textbf{Implicación} & Se encarga ofrecer un contacto entre los clientes y la empresa. \\
\hline
\end{tabular}
\caption{Encargado de oficina}
\end{center}
\end{table}

\subsubsection{Necesidades principales de los implicados}
\begin{table}[H]
	\begin{center}
	\begin{tabular}{|l|l|m{3cm}|m{2.75cm}|m{3.25cm}|}
			\hline
			\textbf{Necesidad} & \textbf{Prioridad} & \textbf{Problema} & \textbf{Solución actual} & \textbf{Solución propuesta }\\
			\hline
			Contratación & Alta & ¿Cómo puedo enviar un paquete? & Ir personalmente a una oficina. & Ofrecer una aplicación web que permita la contratación de los distintos servicios. \\
			\hline
			Consulta & Alta & ¿Cómo puedo saber el estado del servicio contratado? & Ponerse en contacto con la oficina personalmente o por teléfono. & Añadir esta característica a la aplicación web. \\
			\hline
			Consulta & Media & ¿Cómo puedo consultar los servicios que he contratado? & Guardar los recibos de cada uno. & Añadir un portal de consulta a la aplicación. \\
			\hline
			Gestión & Alta & ¿Cómo puedo modificar parámetros de un servicio en curso? & Ponerse en contacto con una oficina. & Añadir esta característica a la aplicación web. \\
			\hline
	\end{tabular}
	\caption{Necesidades por parte de los clientes}
\end{center}
\end{table}

\begin{table}[H]
\begin{center}
	\begin{tabular}{|l|l|m{3cm}|m{2.75cm}|m{3.25cm}|}
\hline
\textbf{Necesidad} & \textbf{Prioridad} & \textbf{Problema} & \textbf{Solución actual} & \textbf{Solución propuesta }\\
\hline
Organización & Alta & ¿Cómo puedo planificar las determinadas rutas que se realizarán en una jornada? & Planearlas a mano. & Diseñar un algoritmo que obtenga la ruta más óptima. \\
\hline
Consulta & Media & ¿Cómo puedo consultar datos relativos a cada oficina/almacén? (número de vehículos, capacidad máxima, etc.) & Llamando por teléfono a cada oficina. & Informatizar la tarea para poder consultar l \\
\hline
Contacto & Media & ¿Cómo puedo contactar con otras oficinas o almacenes? & Por teléfono. & Añadir un chat a la aplicación. \\
\hline
\end{tabular}
\caption{Necesidades por parte de los empleados}
\end{center}
\end{table}


\subsection{Requisitos funcionales}
Requisitos más importantes a nivel funcional que debe incluir el sistema.
\begin{enumerate}[label=\textbf{RF-\arabic*}]
	\item \textbf{Gestión de envíos}\\
	El sistema debe manejar los envíos desde el momento que son prerregistrados hasta la 				entrega de los mismos.
	\begin{enumerate}[label=\textbf{RF-1.\arabic*}]
		\item Monitorizar cada envío con un número de seguimiento que proporcionará información 			que se actualizará cada vez que el paquete pase por una fase determinada.
		\begin{enumerate}[label=\textbf{RF-1.1.\arabic*}]
			\item Dotar a oficinas y repartidores de un sistema para escanear el código de los 					paquetes y actualizar su traza.
		\end{enumerate}
		\item Permitir al destinatario la gestión del envío antes de su entrega.
		\begin{enumerate}[label=\textbf{RF-1.2.\arabic*}]
			\item Cambiar la fecha de entrega.
			\item Autorizar a un vecino en caso de ausencia.
			\item Elegir la franja horaria de recepción más óptima.
		\end{enumerate}
		\item Permitir a la empresa la gestión de las distintas oficinas y almacenes
		\begin{enumerate}[label=\textbf{RF-1.3.\arabic*}]
			\item Establecer las rutas para realizar el reparto de la forma más 								eficiente y eficaz.
			\item Consultar la localización de cada vehículo de la flota.
			\item Consultar los paquetes que se encuentran en cada oficina.
		\end{enumerate}	
	\end{enumerate}
	\item \textbf{Gestión de socios}\\
	Debe haber un control sobre los socios afiliados a la empresa.
		\begin{enumerate}[label=\textbf{RF-2.\arabic*}]
			\item Gestionar el alta de los socios.
			\item Gestionar la baja de los socios.
			\item Obtener información sobre un socio.
				\begin{enumerate}[label=\textbf{RF-2.3.\arabic*}]
					\item Ver datos personales.
					\item Ver tasa de incidencias.
					\item Ver número de envíos realizados al año.
				\end{enumerate}
			\item Modificar datos del socio.
			\item Gestión de pagos.	
				\begin{enumerate}[label=\textbf{RF-2.5.\arabic*}]
					\item Método de pago.
					\item Descuentos aplicados.
					\item Servicios no pagados.
					\item Fecha del próximo pago.
				\end{enumerate}
		\end{enumerate}
	\item \textbf{Gestión de valoraciones}\\ 
	Permitir a los usuarios registrados (socios) valorar el sistema	mediante breves encuestas que 	se realizarán tras cada servicio.	
	\begin{enumerate}[label=\textbf{RF-3.\arabic*}]
			\item Crear estadísticas sobre la valoración general de la empresa.
			\item Métodos para solucionar las incidencias que hayan podido afectar a los socios.
	\end{enumerate}
\end{enumerate}
\subsection{Requisitos no funcionales}
Requisitos relativos a la usabilidad del sistema.
\begin{enumerate}[label=\textbf{RNF-\arabic*}]
	\item Facilidad
		\begin{enumerate}[label=\textbf{RNF-1.\arabic*}]
			\item Ayuda en línea a los potenciales clientes o socios.
			\item Interfaz amigable y rápida.
			\item Aplicación móvil para la contratación o gestión del servicio.
		\end{enumerate}
	\item Fiabilidad
		\begin{enumerate}[label=\textbf{RNF-2.\arabic*}]
			\item Copias de seguridad diarias de la información sensible.
			\item Claves de acceso al sistema con privilegios variables según el tipo de 						empleado.
		\end{enumerate}
	\item Rendimiento
		\begin{enumerate}[label=\textbf{RNF-3.\arabic*}]
			\item Terminales en las oficinas donde los clientes podrán obtener la información 					necesaria y realizar la gestión correspondiente, mejorando la agilidad en las 						oficinas.
			\item Posibilidad de pagar el envío en el propio terminal.
		\end{enumerate}
	\item Físicos
		\begin{enumerate}[label=\textbf{RNF-4.\arabic*}]
			\item Los paquetes deberán llevarse a una oficina o bien ser recogidos por un 						repartidor. En el segundo caso, el cliente deberá imprimir una etiqueta con el código 			de barras.
		\end{enumerate}
		
\end{enumerate}
\subsection{Requisitos de información}
Información que ha de almacenarse en el sistema.
\begin{enumerate}[label=\textbf{RI-\arabic*}]
	\item \textbf{Envíos}\\
	\begin{enumerate}[label=\textbf{RI-1.\arabic*}]
		\item Tarifa del envío.
		\item Lugar y fecha de la recogida.
		\item Datos del emisor.
		\begin{enumerate}[label=\textbf{RI-1.2.\arabic*}]
			\item Nombre.
			\item Dirección.
			\item Teléfono.
		\end{enumerate}
		\item Datos del destinatario.
		\begin{enumerate}[label=\textbf{RI-1.3.\arabic*}]
			\item Nombre.
			\item Dirección.
			\item Teléfono.
			\item Autorizado.
			\item Notas.
		\end{enumerate}
		\item Peso.
		\item Dimensiones.
		\item Indicaciones de seguridad.
		\item Vehículo asociado.
		\item Oficina asociada.
	\end{enumerate}
	\item \textbf{Flota}\\
	\begin{enumerate}[label=\textbf{RI-2.\arabic*}]
		\item Número de vehículos asociados a cada almacén.
		\item Autonomía de cada vehículo.
		\item Ruta de cada vehículo.
		\item Matrícula de cada vehículo.
		\item Conductor de cada vehículo.
	\end{enumerate}
	\item \textbf{Trabajadores}
	\begin{enumerate}[label=\textbf{RI-3.\arabic*}]
		\item Puesto.
		\item Sueldo.
		\item Incidencias.
		\item Período de vacaciones.
		\item Valoración media.
	\end{enumerate}	
	\item \textbf{Oficina}
	\begin{enumerate}[label=\textbf{RI-4.\arabic*}]
		\item Capacidad.
		\item Número de envíos salientes.
		\item Número de envíos entrantes.
	\end{enumerate}		
	\item \textbf{Almacén}
		\begin{enumerate}[label=\textbf{RI-5.\arabic*}]
			\item Capacidad.
			\item Número de mozos de almacén.
		\end{enumerate}	
	\item \textbf{Valoración}
		\begin{enumerate}[label=\textbf{RI-6.\arabic*}]
			\item Valoración media por provincia.
			\item Categoría de las incidencias.
		\end{enumerate}	
	\item \textbf{Socios}
			\begin{enumerate}[label=\textbf{RI-7.\arabic*}]
			\item Tarifas.
			\item Deudas.
			\item Datos fiscales.
			\item Descuentos.
			\item Direcciones de recogida.
			\item Valoración media de la empresa.
			\item Volumen de pedidos.
		\end{enumerate}	
\end{enumerate}

\end{document}