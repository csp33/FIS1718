\usepackage{color}
\usepackage{xcolor}
\usepackage{colortbl}
\usepackage{multirow}





\begin{table}[H]

\centering
\begin{tabular}{|m{3cm}|m{4cm}|m{2cm}|m{2cm}|m{2cm}|m{1cm}|}
\hline
\textbf{Actor} &  \multicolumn{4}{m{8cm}|}{nombre} \vline &  \cellcolor{gray!40}ID \\
\hline
\textbf{Descripción} & \multicolumn{5}{m{8cm}|}{desc} \\
\hline
\textbf{Características} & \multicolumn{5}{m{8cm}|}{carac} \\
\hline
\textbf{Relaciones} &\multicolumn{5}{m{8cm}|}{relac} \\
\hline
\textbf{Referencias} & \multicolumn{5}{m{8cm}|}{ref} \\
\hline
\textbf{Autor} & autor & \textbf{Fecha} & fecha & \textbf{Versión} & v \\
\hline
\end{tabular}

\vspace{1cm}

\begin{tabular}{|m{4cm}|m{7.3cm}|m{4cm}|}
\hline
\multicolumn{3}{|m{15.3cm}|}{\textbf{Atributos}} \\
\hline
\textbf{Nombre} & \textbf{Descripción} & \textbf{Tipo} \\
\hline
n1 & desc1 & tipo1 \\
\hline
\end{tabular}


\vspace{1cm}

\begin{tabular}{|m{16.2cm}|}
\hline
\textbf{Comentarios} \\
\hline
coms \\
\hline
\end{tabular}

\caption{Actor}

\end{table}