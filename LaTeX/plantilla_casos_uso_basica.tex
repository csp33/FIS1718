\usepackage[table]{xcolor}


\begin{table}[H]
\centering
\begin{tabular}{|m{3cm}|m{2.65cm}|m{2cm}|m{2cm}|m{2cm}|m{1cm}|}
\hline
\textbf{Caso de uso} &  \multicolumn{4}{m{9.65cm}|}{\textit{Nombre del caso de uso}} &  \cellcolor{gray!40}ID \\
\hline
\textbf{Actores} & \multicolumn{5}{m{9.65cm}|}{\textit{Actores que participan en el caso de uso, indicando si son principales o secundarios.}} \\
\hline
\textbf{Tipo} & \multicolumn{5}{m{9.65cm}|}{\textit{Tipo de caso de uso.}} \\
\hline
\textbf{Referencias} &\multicolumn{3}{m{4.77cm}|}{\textit{Requisitos funcionales}} & \multicolumn{2}{m{4.77cm}|}{\textit{Casos de uso}} \\
\hline
\textbf{Precondición} & \multicolumn{5}{m{9.65cm}|}{\textit{Condiciones sobre el estado del sistema que deben cumplirse para que se pueda realizar el caso de uso.}} \\
\hline
\textbf{Postcondición} & \multicolumn{5}{m{9.65cm}|}{\textit{Efectos que tiene la realización del caso de uso sobre el sistema.}} \\
\hline
\textbf{Autor} &  & \textbf{Fecha} &  & \textbf{Versión} &  \\
\hline
\end{tabular}

\vspace{1cm}

\begin{tabular}{|m{16.2cm}|}
\hline
\textbf{Propósito} \\
\hline
\textit{Descripción del objetivo que cubre el caso de uso (una línea).} \\
\hline
\end{tabular}

\vspace{1cm}

\begin{tabular}{|m{16.2cm}|}
\hline
\textbf{Resumen} \\
\hline
\textit{Descripción a alto nivel de la secuencia de acciones realizadas en el caso de uso (un párrafo).} \\
\hline
\end{tabular}

\caption{Plantilla básica para casos de uso}

\end{table}

